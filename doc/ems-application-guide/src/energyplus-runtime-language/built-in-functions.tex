\section{Built-In Functions}\label{built-in-functions}

Several useful, built-in functions are available for use in Erl programs. You cannot configure these; they are internal to the language processor inside EnergyPlus. They provide access to a subset of general service routines that are useful inside the main EnergyPlus program or are intrinsic functions available in the underlying Fortran language. The ``@'' character is used to signal to the language processor that the following character string defines a built-in function that is used to assign a result to an Erl variable. The characters appended to the ``@'' operator must be one of the predefined names listed in Table~\ref{table:built-in-math-functions-for-erl}, Table~\ref{table:built-in-energyplus-simulation-management}, Table~\ref{table:built-in-functions-for-trend-variables-in-erl}, or Table~\ref{table:built-in-psychrometric-functions-for-erl}. The syntax of the function call will vary depending on the arguments required by the function, but the general structure is:

SET \textless{}variable\textgreater{} = @\textless{}function name\textgreater{}~ \textless{}argument1\textgreater{}~ \textless{}argument2\textgreater{}~~ \ldots{}~~ \textless{}argumentN\textgreater{}

Where ``argument'' can be either an Erl variable or a numeric constant.

For example, the following two statements can be used to set the value of an Erl variable called mySupplyRH to have percent relative humidity.

SET mySupplyRH = @RhFnTdbWPb mySupplyDryblub mySupplyHumRat mySupplyPress

SET mySupplyRH = mySupplyRH * 100

\subsection{Built-in Math Functions}\label{built-in-math-functions}

Table~\ref{table:built-in-math-functions-for-erl} lists the built-in functions for common mathematical functions. The numerical model for these functions is provided by the underlying Fortran language and the compiler.

% table 4
\begin{longtable}[c]{p{1.5in}p{3.0in}p{1.5in}}
\caption{Built-in Math Functions for Erl \label{table:built-in-math-functions-for-erl}} \tabularnewline
\toprule
Function Name & Description & Number of Arguments \tabularnewline
\midrule
\endfirsthead

\caption[]{Built-in Math Functions for Erl} \tabularnewline
\toprule
Function Name & Description & Number of Arguments \tabularnewline
\midrule
\endhead

@Round & Decreases precision of real number argument to nearest whole number, remains a real number. & 1 \tabularnewline
@Mod & Returns remainder after dividing the first argument by the second. & 2 \tabularnewline
@Sin & Sine, returns sine of angle given in radians. & 1 \tabularnewline
@Cos & Cosine, returns cosine of angle given in radians. & 1 \tabularnewline
@ArcSin & Arcsine, returns angle in radians from sine of angle. & 1 \tabularnewline
@ArcCos & ArcCosine, returns angle in radians from cosine of angle. & 1 \tabularnewline
@DegToRad & Degrees to radians, returns radians from degrees. & 1 \tabularnewline
@RadToDeg & Radians to degrees, returns degrees from radians. & 1 \tabularnewline
@Exp & Exponential, e  , returns result. & 1 \tabularnewline
@Ln & Natural log, log  (x), returns result. & 1 \tabularnewline
@Max & Maximum, returns largest value of two arguments. & 2 \tabularnewline
@Min & Minimum, returns smallest value of two arguments. & 2 \tabularnewline
@Abs & Absolute value, returns positive magnitude of argument. & 1 \tabularnewline
@RandomUniform & Pseudo-Random Number Generator, returns random number with uniform probability distribution across the range of values passed as the arguments, inclusive. Argument 1 is the lower limit. Argument 2 is the upper limit. & 2 \tabularnewline
@RandomNormal & Pseudo-Random Number Generator, returns random number with normal (Gaussian) probability distribution as a function of the mean, standard deviation, and limits. Argument 1 is the mean. Argument 2 is the standard deviation. Argument 3 is the lower limit. Argument 4 is the upper limit. & 4 \tabularnewline
@SeedRandom & Random Seed, controls the seed used in the random number generator for calls to @RandomUniform and @RandomNormal.~ Use is optional and provided for repeatable series of random numbers. The argument is rounded to the nearest whole number and then used to set the size and values of the seed for the number generator.~ & 1 \tabularnewline
\bottomrule
\end{longtable}

\subsection{Built-In Simulation Management Functions}\label{built-in-simulation-management-functions}

% table 5
\begin{longtable}[c]{p{1.5in}p{3.0in}p{1.5in}}
\caption{Built-in EnergyPlus Simulation Management Functions for Erl \label{table:built-in-energyplus-simulation-management}} \tabularnewline
\toprule
Function Name & Description & Number of Arguments \tabularnewline
\midrule
\endfirsthead

\caption[]{Built-in EnergyPlus Simulation Management Functions for Erl} \tabularnewline
\toprule
Function Name & Description & Number of Arguments \tabularnewline
\midrule
\endhead

@FatalHaltEp & Throws fatal error with time of occurrence and stops execution of current model. Argument passes a number that can be used as an error code. & 1 \tabularnewline
@SevereWarnEp & Throws severe error with time of occurrence and continues execution. Argument passes a number that can be used as an error code. & 1 \tabularnewline
@WarnEp & Throws warning error and continues execution. Argument passes a number that can be used as an error code. & 1 \tabularnewline
\bottomrule
\end{longtable}

\subsection{Built-in Trend Variable Functions}\label{built-in-trend-variable-functions}

For control algorithms, you often need to be able put a sensor reading into some historical context. The trend variables are provided in Erl as a way to log the time history of data to use in control decisions. To use the trend variables in Erl programs, their values must be extracted and placed into normal Erl variables. Setting up an Erl variable as a trend variable requires an EnergyManagementSystem:TrendVariable input object. The access functions listed in Table~\ref{table:built-in-functions-for-trend-variables-in-erl} are used to obtain data from a trend variable during the execution of an Erl program. These functions act on trend variables and return values into the user's Erl variables for subsequent use in calculations. Each trend function takes the name of the trend variable and an index that identifies how far back in time the function should be applied. Trend variable names are also Erl variables but with special pointers to another data structure with the time series data storage. The trend logs have a first-in, first-out storage array where only the most recent data are retained. Each element in the history corresponds to the result for that value over a zone timestep. The time difference between trend log items is the zone timestep in hours, so that the slope returned by @TrendDirection is in per-hour units.

% table 6
\begin{longtable}[c]{p{1.5in}p{3.0in}p{1.5in}}
\caption{Built-in Functions for Trend Variables in Erl \label{table:built-in-functions-for-trend-variables-in-erl}} \tabularnewline
\toprule
Function Name & Description & Number of Arguments \tabularnewline
\midrule
\endfirsthead

\caption[]{Built-in Functions for Trend Variables in Erl} \tabularnewline
\toprule
Function Name & Description & Number of Arguments \tabularnewline
\midrule
\endhead

@TrendValue & Returns history value for a particular number of timesteps into the past. Dereferences data stored in trend into another Erl variable. Takes trend variable name and the specific timestep into the past to return. & 2 \tabularnewline
@TrendAverage & Returns historical average (mean) for values in trend variable. Takes trend variable name and number of steps into the past to analyze & 2 \tabularnewline
@TrendMax & Returns historical maximum for values in trend variable. Takes trend variable name and number of steps into the past to analyze. & 2 \tabularnewline
@TrendMin & Returns historical minimum for values in trend variable within the index. Takes trend variable name and number of steps into the past to analyze. & 2 \tabularnewline
@TrendDirection & Returns slope of a linear least squares fit of trend data within the index. Positive if trend is increasing, negative if decreasing. Takes trend variable name and number of steps into the past to analyze. & 2 \tabularnewline
@TrendSum & Returns sum of elements stored in trend. Takes trend variable name and number of steps into the past to analyze. & 2 \tabularnewline
\bottomrule
\end{longtable}

The trend functions all take as their second argument an array index. This number should be considered an integer for locating an array position. (It will be rounded down to the nearest integer using Fortran's FLOOR intrinsic.)~ This index argument tells the trend functions how far back in time they should reach into the history log when they evaluate the function call. This enables you to compare long- and short-range trends.

\subsection{Built-in Psychrometric Functions}\label{built-in-psychrometric-functions}

Building modeling often involves calculations related to moist air. A comprehensive set of built-in functions is available for psychrometric calculations. Table~\ref{table:built-in-psychrometric-functions-for-erl} lists the functions available for use in Erl programs that are related to moist air properties and some physical properties related to water. More discussion of the psychrometric functions is provided in the section ``Pyschrometric services'' in the Module Developer Guide.

% table 7
\begin{longtable}[c]{p{1.5in}p{1.5in}p{1.5in}p{1.5in}}
\caption{Built-in Psychrometric Functions for Erl \label{table:built-in-psychrometric-functions-for-erl}} \tabularnewline
\toprule
Function Name & Arguments & Description & Units \tabularnewline
\midrule
\endfirsthead

\caption[]{Built-in Psychrometric Functions for Erl} \tabularnewline
\toprule
Function Name & Arguments & Description & Units \tabularnewline
\midrule
\endhead

@RhoAirFnPbTdbW & Result & Density of moist air & \si{\kg\per\m\cubed} \tabularnewline
Input 1 & Barometric pressure & Pa \tabularnewline
Input 2 & Drybulb temperature & °C \tabularnewline
Input 3 & Humidity ratio & kgWater/kgDryAir \tabularnewline
@CpAirFnW & Result & Heat capacity of moist air & J/kg-°C \tabularnewline
Input 1 & Humidity ratio & kgWater/kgDryAir \tabularnewline
@HfgAirFnWTdb` & Result & Heat of vaporization for vapor & J/kg \tabularnewline
Input 1 & Humidity ratio & kgWater/kgDryAir \tabularnewline
Input 2 & Drybulb temperature & °C \tabularnewline
@HgAirFnWTdb & Result & Enthalpy of the gas &  \tabularnewline
Input 1 & Humidity ratio & kgWater/kgDryAir \tabularnewline
Input 2 & Drybulb temperature & °C \tabularnewline
@TdpFnTdbTwbPb & Result & Dew-point temperature & °C \tabularnewline
Input 1 & Drybulb temperature & °C \tabularnewline
Input 2 & Wetbulb temperature & °C \tabularnewline
Input 3 & Barometric pressure & Pa \tabularnewline
@TdpFnWPb & Result & Dew-point temperature & °C \tabularnewline
Input 1 & Humidity ratio & kgWater/kgDryAir \tabularnewline
Input 2 & Barometric pressure & Pa \tabularnewline
@HFnTdbW & Result & Enthalpy of moist air & J/kg \tabularnewline
Input 1 & Drybulb temperature & °C \tabularnewline
Input 2 & Humidity ratio & kgWater/kgDryAir \tabularnewline
@HFnTdbRhPb & Result & Enthalpy of moist air & J/kg \tabularnewline
Input 1 & Drybulb temperature & °C \tabularnewline
Input 2 & Relative humidity & Fraction (0.0 .. 1) \tabularnewline
Input 3 & Barometric pressure & Pa \tabularnewline
@TdbFnHW & Result & Drybulb temperature & °C \tabularnewline
Input 1 & Enthalpy of moist air & J/kg \tabularnewline
Input 2 & Humidity ratio & kgWater/kgDryAir \tabularnewline
@RhovFnTdbRh & Result & Vapor density in air & \si{\kg\per\m\cubed} \tabularnewline
Input 1 & Drybulb temperature & °C \tabularnewline
Input 2 & Relative humidity & Fraction (0.0 .. 1) \tabularnewline
@RhovFnTdbWPb & Result & Vapor density in air & \si{\kg\per\m\cubed} \tabularnewline
Input 1 & Drybulb temperature & °C \tabularnewline
Input 2 & Humidity ratio & kgWater/kgDryAir \tabularnewline
Input 3 & Barometric pressure & Pa \tabularnewline
@RhFnTdbRhov & Result & Relative humidity & Fraction (0.0 .. 1) \tabularnewline
Input 1 & Drybulb temperature & °C \tabularnewline
Input 2 & Vapor density in air & \si{\kg\per\m\cubed} \tabularnewline
@RhFnTdbWPb & Result & Relative humidity & Fraction (0.0 .. 1) \tabularnewline
Input 1 & Drybulb temperature & °C \tabularnewline
Input 2 & Humidity ratio & kgWater/kgDryAir \tabularnewline
Input 3 & Barometric pressure & Pa \tabularnewline
@TwbFnTdbWPb & Result & Wetbulb temperature & °C \tabularnewline
Input 1 & Drybulb temperature & °C \tabularnewline
Input 2 & Humidity ratio & kgWater/kgDryAir \tabularnewline
Input 3 & Barometric pressure & Pa \tabularnewline
@VFnTdbWPb & Result & Specific volume & \si{\m\cubed}/kg \tabularnewline
Input 1 & Drybulb temperature & °C \tabularnewline
Input 2 & Humidity ratio & kgWater/kgDryAir \tabularnewline
Input 3 & Barometric pressure & Pa \tabularnewline
@WFnTdpPb & Result & Humidity ratio & kgWater/kgDryAir \tabularnewline
Input 1 & Dew-point temperature & °C \tabularnewline
Input 2 & Barometric pressure & Pa \tabularnewline
@WFnTdbH & Result & Humidity ratio & kgWater/kgDryAir \tabularnewline
Input 1 & Drybulb temperature & °C \tabularnewline
Input 2 & Enthalpy of moist air & J/kg \tabularnewline
@WFnTdbTwbPb & Result & Humidity ratio & kgWater/kgDryAir \tabularnewline
Input 1 & Drybulb temperature & °C \tabularnewline
Input 2 & Wetbulb temperature & °C \tabularnewline
Input 3 & Barometric pressure & Pa \tabularnewline
@WFnTdbRhPb & Result & Humidity ratio & kgWater/kgDryAir \tabularnewline
Input 1 & Drybulb temperature & °C \tabularnewline
Input 2 & Relative humidity & Fraction (0.0 .. 1) \tabularnewline
Input 3 & Barometric pressure & Pa \tabularnewline
@PsatFnTemp & Result & Saturation pressure & Pa \tabularnewline
Input 1 & Drybulb temperature & °C \tabularnewline
@TsatFnHPb & Result & Saturation temperature & °C \tabularnewline
Input 1 & Enthalpy of moist air & J/kg \tabularnewline
Input 2 & Barometric pressure & Pa \tabularnewline
@CpCW & Result & Heat capacity of water & J/kg \tabularnewline
Input 1 & Temperature & °C \tabularnewline
@CpHW & Result & Heat capacity of water & J/kg \tabularnewline
Input 1 & Temperature & °C \tabularnewline
@RhoH2O & Result & Density of water & \si{\kg\per\m\cubed} \tabularnewline
Input 1 & Temperature & °C \tabularnewline
\bottomrule
\end{longtable}

\subsection{Built-in Curve and Table Functions}\label{built-in-curve-and-table-functions}

EnergyPlus has a number of different generic curve and table input objects that are used to describe the performance characteristics for various component models.~ Table~\ref{table:built-in-function-for-accessing-curves} describes a built-in function called @CurveValue that is available for reusing those curve and table input objects in your Erl programs.~ Although the Erl language could be used to replicate the functionality, reusing those input objects can have advantages because the input may have already been developed for use in traditional component models or the limiting and interpolation methods are helpful.~ The @CurveValue function expects six arguments, although usually only a subset of them will be used depending on the number of independent variables involved with the curve or table.~ Because Erl does not support passing optional arguments, dummy variables do need to be included in the function call for all unused independent variables.~   For example, the Curve:Biquadratric object has only x and y independent variables, so input arguments 4, 5, and 6 will not be used when @CurveValue is evaluated:

~~~~~ Set MyCurveResult = @CurveValue myCurveIndex X1 Y1 dummy dummy dummy;

Note that although version 8.6 of EnergyPlus introduced changes to not allow uninitialized variables in expressions, @CurveValue has an exception to this for backward compatibility.  @CurveValue only issues errors to the EDD file and does not fatal when called with uninitialized dummy variables.

The first input argument is always an Erl variable that has been declared using an EnergyManagementSystem:CurveOrTableIndexVariable input object.~ This variable identifies the location of a specific curve or table in the program's internal data structures.~ It is important that you do not inadvertently reassign the value held in this variable because it is only filled once at the beginning of the simulation.

% table 8
\begin{longtable}[c]{p{1.5in}p{1.5in}p{1.5in}p{1.5in}}
\caption{Built-in Function for Accessing Curves and Tables \label{table:built-in-function-for-accessing-curves}} \tabularnewline
\toprule
Function Name & Arguments & Description & Notes \tabularnewline
\midrule
\endfirsthead

\caption[]{Built-in Function for Accessing Curves and Tables} \tabularnewline
\toprule
Function Name & Arguments & Description & Notes \tabularnewline
\midrule
\endhead

@CurveValue & Result & Result from evaluating the curve or table as a function of the input arguments &  \tabularnewline
Input 1 & Index variable that "points" to a specific curve or table object defined elsewhere in the IDF. & This variable needs to be declared and filled using an EnergyManagementSystem:CurveOrTableIndexVariable object. \tabularnewline
Input 2 & First independent variable & Typically the "X" input value, always used \tabularnewline
Input 3 & Second independent variable & Typically the "Y" value, only used if curve/table has two or more independent variables \tabularnewline
Input 4 & Third independent variable & Typically the "Z" value, only used if curve/table has three or more independent variables. \tabularnewline
Input 5 & Fourth independent variable & Only used if table has four or more independent variables \tabularnewline
Input 6 & Fifth independent variable & Only used if table has five independent variables \tabularnewline
\bottomrule
\end{longtable}

\subsection{Built-in Weather Data Functions}\label{built-in-weather-data-functions}

EnergyPlus sets up 24 hours of weather data for Today and Tomorrow during DesignDay setup or based on incoming data from the weather file (epw). This data is used as the basis for setting the environment variables for a given zone timestep in WeatherManager::SetCurrentWeather. This data can be useful for predictive control or for setting Weather Data actuator overrides based on the incoming weather data using the BeginZoneTimestepBeforeSetCurrentWeather calling point.

Table~\ref{table:built-in-function-for-weather-data} describes a set of built-in functions to access this weather data. For all of these functions, the first argument is the hour (from 0 to 23) and the second argument is the zone timestep number (from 1 to number of timesteps per hour). To access the current timestep use internal variables ``Hour''' and ``TimeStepNum'':
\begin{lstlisting}
@TodayBeamSolarRad Hour TimeStepNum,
\end{lstlisting}

% table 9
\begin{longtable}[c]{p{2.25in}p{2.75in}p{1.0in}}
\caption{Built-in Functions for Accessing Today and Tomorrow Weather Data \label{table:built-in-function-for-weather-data}} \tabularnewline
\toprule
Function Name & Description & Units \tabularnewline
\midrule
\endfirsthead

\caption[]{Built-in Psychrometric Functions for Erl} \tabularnewline
\toprule
Function Name & Description & Units \tabularnewline
\midrule
\endhead

@TodayIsRain & Rain indicator, 1.0 = raining &  \tabularnewline
@TodayIsSnow & Snow indicator, 1.0 = snow on ground &  \tabularnewline
@TodayOutDryBulbTemp & Outdoor dry-bulb temperature & \SI{}{\degreeCelsius} \tabularnewline
@TodayOutDewPointTemp & Outdoor dewpoint temperature & \SI{}{\degreeCelsius} \tabularnewline
@TodayOutBaroPress & Outdoor barometric pressure & \si{pascal} \tabularnewline
@TodayOutRelHum & Outdoor relative humidity & \si{percent} \tabularnewline
@TodayWindSpeed & Wind speed & \si{\meter\per\second} \tabularnewline
@TodayWindDir & Wind direction (N = 0, E = 90, S = 180, W = 270) & degrees \tabularnewline
@TodaySkyTemp & Sky temperature & \SI{}{\degreeCelsius} \tabularnewline
@TodayHorizIRSky & Horizontal infrared radiation rate per area & \si{W\per\meter\squared} \tabularnewline
@TodayBeamSolarRad & Direct normal solar irradiance & \si{W\per\meter\squared} \tabularnewline
@TodayDifSolarRad & Diffuse horizontal solar irradiance & \si{W\per\meter\squared} \tabularnewline
@TodayAlbedo & Ratio of ground reflected solar to global horizontal irradiance (unused) & dimensionless \tabularnewline
@TodayLiquidPrecip & Liquid precipitation depth & \si{\mm} \tabularnewline
@TomorrowIsRain & Rain indicator, 1.0 = raining &  \tabularnewline
@TomorrowIsSnow & Snow indicator, 1.0 = snow on ground &  \tabularnewline
@TomorrowOutDryBulbTemp & Outdoor dry-bulb temperature & \SI{}{\degreeCelsius} \tabularnewline
@TomorrowOutDewPointTemp & Outdoor dewpoint temperature & \SI{}{\degreeCelsius} \tabularnewline
@TomorrowOutBaroPress & Outdoor barometric pressure & \si{pascal} \tabularnewline
@TomorrowOutRelHum & Outdoor relative humidity & \si{percent} \tabularnewline
@TomorrowWindSpeed & Wind speed & \si{\meter\per\second} \tabularnewline
@TomorrowWindDir & Wind direction (N = 0, E = 90, S = 180, W = 270) & degrees \tabularnewline
@TomorrowSkyTemp & Sky temperature & \SI{}{\degreeCelsius} \tabularnewline
@TomorrowHorizIRSky & Horizontal infrared radiation rate per area & \si{W\per\meter\squared} \tabularnewline
@TomorrowBeamSolarRad & Direct normal solar irradiance & \si{W\per\meter\squared} \tabularnewline
@TomorrowDifSolarRad & Diffuse horizontal solar irradiance & \si{W\per\meter\squared} \tabularnewline
@TomorrowAlbedo & Ratio of ground reflected solar to global horizontal irradiance (unused) & dimensionless \tabularnewline
@TomorrowLiquidPrecip & Liquid precipitation depth & \si{\mm} \tabularnewline
\bottomrule
\end{longtable}
