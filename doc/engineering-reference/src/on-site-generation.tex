\chapter{On-Site Generation, Power Conversion, and Storage}\label{on-site-generation}

This section discusses on-site generation, electric storage, and power conversion models related to serving the facility with electric power.  EnergyPlus is usually predicting the electric energy consumed in a building and its HVAC equipment over time.  If this electric power all comes directly from the grid and the building has no power conversion devices (such as a transformer, inverter, battery, generators, solar panels etc.) then only some basic accounting is needed and none of the additional models described here are really needed.  The device models in this section are for when the electric power service system supplying the facility has some added equipment that is considered part of the building being modeled.   The purpose of electric power service modeling in EnergyPlus is to account for the energy performance of such equipment at each timestep of the building and HVAC simulation.  The building and HVAC simulation provides a current prediction of the electric load required by the facility and the electric power service simulation uses that information.   This is simplified power system model with the only goal being to account for power losses (in Watts) and energy consumed (in Joules) by key devices in the balance of system including:

\begin{description}
  \item[Transformer]: a building may have transformers and they are not 100\% efficient. A facility may own its own transformer and have high voltage utility grid connection. An on-site generation system might have an isolation transformer.  The EnergyPlus model meters electric losses as a function of the power going through each transformer.
  \item[Fuel-fired generators]: a building may have generators that consume one type of energy resource (natural gas, diesel, etc.) and generate electricity.  These generators need supervisory control to determine when and how hard to run the engine.  The generators may be used for cogeneration and include connections to a hot water plant in the HVAC models.  The generator models determine the efficiency as fuels are converted to electric power and heat. 
  \item[Renewable generators]: a building may have wind turbines or solar photovoltaic panels.  The generator models determine the efficiency as wind and sun are converted to electricity.
  \item[Inverters]: the direct current generated by solar photovoltaic or discharged from storage may need to be converted from DC to AC by an inverter.  The EnergyPlus models meter electric losses as a function of the power going through each inverter.
  \item[Storage]: a building may have on-site electric storage and it is not 100\% efficient.  Storage can be used to adjust utility demand profile or store on-site renewables.
\end{description}

There is no explicit modeling of voltage and current between devices, just basic power and energy exchanges.  There is no modeling of power loss in conductors and distribution panels throughout the building.  There is no modeling of short term power draws as for a motor start -- power levels are modeled as single-valued averages for the entire system timestep (1 minute or longer).