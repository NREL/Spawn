\section{Introduction}\label{introduction}

EnergyPlus is a simulation program designed for modeling buildings with all their associated heating, ventilating, and air conditioning equipment. EnergyPlus is a simulation \emph{engine}: it was designed to be an element within a system of programs that would include a graphical user interface to describe the building. However, it can be run \emph{stand alone} without such an interface. This document describes how to run EnergyPlus in such a stand alone fashion. This section will introduce you to the EP-Launch program, which helps you run EnergyPlus.~ EP-Launch looks and acts pretty much like a standard Windows\textsuperscript{TM} program, so if you just want to get started with some exercises, you can skip to the section ``Tutorial Example for running EnergyPlus'' first and come back to this section if you run into problems with EP-Launch.

Like all simulation programs, EnergyPlus consists of more than just an executable file. EnergyPlus needs various input files that describe the building to be modeled and the environment surrounding it. The program produces several output files, which need to be described or further processed in order to make sense of the results of the simulation. Finally, even in stand-alone mode, EnergyPlus is usually not executed ``by hand'', but rather by running a procedure file which takes care of finding input files and storing or further processing the output files.

To assist those in the Windows environment, we have included the EP-Launch program. Review the next section for basic instructions. More advanced techniques of executing the program are contained in the Auxiliary Programs document under ``Technical Details of running EnergyPlus'', including some advanced uses of the EP-Launch program. If you wish to learn about DOS/Command Line use for EnergyPlus, you will need to read that section in the Auxiliary Programs document.
