\documentclass[12pt,twoside,letterpaper,titlepage]{report}

\usepackage{amssymb,amsmath}

% By default mathspec uses no-math, which is passed to fontspec, and creates problem with siunitx
\usepackage[math]{mathspec}

\defaultfontfeatures{Ligatures=TeX,Scale=MatchLowercase}
\usepackage[margin=0.75in, top=1in]{geometry}
% fixltx2e has been merged into LaTeX2e proper
% see: https://latex-project.org/ltnews/ltnews22.pdf
% \usepackage{fixltx2e}
\usepackage{graphicx,grffile}
\makeatletter
\def\maxwidth{\ifdim\Gin@nat@width>\linewidth\linewidth\else\Gin@nat@width\fi}
\def\maxheight{\ifdim\Gin@nat@height>\textheight\textheight\else\Gin@nat@height\fi}
\makeatother
% Scale images if necessary, so that they will not overflow the page
% margins by default, and it is still possible to overwrite the defaults
% using explicit options in \includegraphics[width, height, ...]{}
\setkeys{Gin}{width=\maxwidth,height=\maxheight,keepaspectratio}
\usepackage[table,svgnames]{xcolor}
\usepackage{framed}
\usepackage{longtable,booktabs}
\usepackage{multirow}
\usepackage{listings}
\usepackage{enumitem}
\PassOptionsToPackage{hyphens}{url}\usepackage[breaklinks=true]{hyperref}

% Background Color for Note/Admonition Boxes
\definecolor{shadecolor}{rgb}{0.9,0.9,0.9}

% Band the Color of Rows in Tables
% unfortunately, doesn't seem to work across the board. Turning off for now
%\rowcolors{1}{white}{lightgray!25!white}

\newenvironment{callout}{\begin{shaded*}\small}{\end{shaded*}}
\newcommand{\warning}[1]{\small{\textbf{\textcolor[rgb]{0.8,0.2,0.2}{#1}}}}

% Font Settings
%\setmainfont[]{Calibri}
%\setmonofont[Mapping=tex-ansi]{Inconsolata}
%\usepackage{newtxtext,newtxmath}

% The depth of numbering for sections
\setcounter{secnumdepth}{4}

% Listing Settings
\lstset{
  backgroundcolor=\color{BlanchedAlmond!20!white},
  basicstyle=\ttfamily\scriptsize,
  breakatwhitespace=false,
  breaklines=true,
  frame=bottomline,
  keepspaces=true,
}

% An environment for describing variables in equations
\newcommand*\wherelistlabel[1]{\ensuremath{#1 =}}
\newenvironment{wherelist}
  {\begin{list}
    {}
    {\addtolength{\leftmargin}{2em}
     \setlength{\itemsep}{0pt}
     \let\makelabel=\wherelistlabel
    }
  }
{\end{list} }

% Provide better table spacing
% From https://www.inf.ethz.ch/personal/markusp/teaching/guides/guide-tables.pdf
\renewcommand{\arraystretch}{1.2} 

\graphicspath{{media/}}

\pagestyle{headings}

% Define custom commands from Pandoc
\setlength{\emergencystretch}{3em}  % prevent overfull lines
\providecommand{\tightlist}{%
  \setlength{\itemsep}{0pt}\setlength{\parskip}{0pt}}

% Redefine (sub)paragraphs to behave more like sections
\ifx\paragraph\undefined\else
\let\oldparagraph\paragraph
\renewcommand{\paragraph}[1]{\oldparagraph{#1}\mbox{}}
\fi
\ifx\subparagraph\undefined\else
\let\oldsubparagraph\subparagraph
\renewcommand{\subparagraph}[1]{\oldsubparagraph{#1}\mbox{}}
\fi

% this is a great package for printing units in a standard way, we should move this way
\usepackage[per-mode=symbol]{siunitx}

% Note: To use a unit without a number, use either `\si{\wattperVolumeFlowRate}`
% or `\SI{}{\wattperVolumeFlowRate}`.

% add some helper unit macros
\DeclareSIUnit\area{\m\squared}
\DeclareSIUnit\volume{\m\cubed}
\DeclareSIUnit\volumeFlowRate{\m\cubed\per\s}
\DeclareSIUnit\massFlowRate{\kg\per\s}
\DeclareSIUnit\density{\kg\per\m\cubed}
\DeclareSIUnit\humidityRatio{\kg\of{W}\per\kg\of{DA}}
\DeclareSIUnit\specificHeatCapacity{\J\per\kg\per\K}
\DeclareSIUnit\specificEnthalpy{\J\per\kg}
\DeclareSIUnit\coefficientOfPerformance{\watt\per\watt}
\DeclareSIUnit\wattperVolumeFlowRate{\watt\s\per\m\cubed} % Displays "W s/m^3"
\DeclareSIUnit\volumeFlowRateperArea{\volumeFlowRate\per\area} % Unused
\DeclareSIUnit\volumeFlowRateperWatt{\volumeFlowRate\per\watt} % Displays "m^3/(s W)"
\DeclareSIUnit\umolperAreaperSecond{\umol\per\area\per\s} % Displays "µmol/(m^2 s)"
\DeclareSIUnit\evapotranspirationRate{\kg\per\area\per\s} % Displays "kg / (m^2 s)"

% Make alias, so its clear these are IP units
\let\DeclareIPUnit\DeclareSIUnit
\let\IP\SI
\let\ip\si

% Add a few IP units, eg use like \IP{80}{\fahrenheit}
\DeclareIPUnit[number-unit-product = \,]
\fahrenheit{ \ensuremath { { } ^ { \circ } } \kern -\scriptspace F }
\DeclareIPUnit\in{in}
\DeclareIPUnit\ft{ft}
\DeclareIPUnit\sqft{\ft\squared}
\DeclareIPUnit\cfm{\ft\cubed\per\minute}
\DeclareIPUnit\CFM{CFM}
\DeclareIPUnit\gal{gal}
\DeclareIPUnit\gpm{gpm}
\DeclareIPUnit\MBH{MBH}


% The current implementation of \degree doesn't use ensuremath and doesn't show
% see https://github.com/josephwright/siunitx/issues/351
% But requried mathspec with options 'math' above fixes it
%\DeclareSIUnit[number-unit-product = \,]
%\degree{ \ensuremath { { } ^ { \circ } } }

% Add a few IP units, eg use like \SI{80}{\fahrenheit}
\DeclareSIUnit[number-unit-product = \,]
\fahrenheit{ \ensuremath { { } ^ { \circ } } \kern -\scriptspace F }

% some additional macros
\newcommand{\PB}[1]{\left(#1\right)}
\newcommand{\RB}[1]{\left[#1\right]}
\newcommand{\CB}[1]{\left\{#1\right\}}

\author{U.S. Department of Energy}
\date{September 30, 2016}
