\section{Group -- Schedules}\label{group-schedules}

This group of objects allows the user to influence scheduling of many items (such as occupancy density, lighting, thermostatic controls, occupancy activity). In addition, schedules are used to control shading element density on the building.

EnergyPlus schedules consist of three pieces: a day description, a week description, and an annual description. An optional element is the schedule type. Each description level builds off the previous sub-level. The day description is simply a name and the values that span the 24 hours in a day to be associated with that name. The week description also has an identifier (name) and twelve additional names corresponding to previously defined day descriptions. There are names for each individual day of the week plus holiday, summer design day, winter design day and two more custom day designations. Finally, the annual schedule contains an identifier and the names and FROM-THROUGH dates of the week schedules associate with this annual schedule. The annual schedule can have several FROM-THROUGH date pairs. One type of schedule reads the values from an external file to facilitate the incorporation of monitored data or factors that change throughout the year.

Schedules are processed by the EnergyPlus Schedule Manager, stored within the Schedule Manager and are accessed through module routines to get the basic values (timestep, hourly, etc). Values are resolved at the Zone Timestep frequency and carry through any HVAC timesteps.

\subsection{Day Type}\label{day-type}

A brief description of ``Day Type'' which is used in the SizingPeriod objects, \hyperref[runperiodcontrolspecialdays]{RunPeriodControl:SpecialDays} object, the Sizing objects and also used by reference in the \hyperref[scheduleweekdaily]{Schedule:Week:Daily} object (discussed later in this section).

Schedules work on days of the week as well as certain specially designated days. Days of the week are the normal -- Sunday, Monday, Tuesday, Wednesday, Thursday, Friday and Saturday. Special day types that can be designated are: Holiday, SummerDesignDay, WinterDesignDay, CustomDay1, CustomDay2. These day types can be used at the user's convenience to designate special scheduling (e.g.~lights, electric equipment, set point temperatures) using these days as reference.

For example, a normal office building may have normal ``occupancy'' rules during the weekdays but significantly different use on weekend. For this, you would set up rules/schedules based on the weekdays (Monday through Friday, in the US) and different rules/schedules for the weekend (Saturday and Sunday, in the US). However, you could also specially designate SummerDesignDay and WinterDesignDay schedules for sizing calculations. These schedules can be activated by setting the Day Type field in the Design Day object to the appropriate season (\textbf{SummerDesignDay} for cooling design calculations; \textbf{WinterDesignDay} for heating design calculations).

In a different building, such as a theater/playhouse, the building may only have occupancy during certain weeks of the year and/or certain hours of certain days. If it was every week, you could designate the appropriate values during the ``regular'' days (Sunday through Saturday). But this would also be an ideal application for the ``CustomDay1'' and/or ``CustomDay2''. Here you would set the significant occupancy, lighting, and other schedules for the custom days and use unoccupied values for the normal weekdays. Then, using a weather file and setting special day periods as appropriate, you will get the ``picture'' of the building usage during the appropriate periods.

\subsection{ScheduleTypeLimits}\label{scheduletypelimits}

Schedule types can be used to validate portions of the other schedules. Hourly day schedules, for example, are validated by range---minimum/maximum (if entered)---as well as numeric type (continuous or discrete). Annual schedules, on the other hand, are only validated for range---as the numeric type validation has already been done.

\subsubsection{Inputs}\label{inputs-042}

\paragraph{Field: Name}\label{field-name-041}

This alpha field should contain a unique (within the schedule types) designator. It is referenced wherever Schedule Type Limits Names can be referenced.

\paragraph{Field: Lower Limit Value}\label{field-lower-limit-value}

In this field, the lower (minimum) limit value for the schedule type should be entered. If this field is left blank, then the schedule type is not limited to a minimum/maximum value range.

\paragraph{Field: Upper Limit Value}\label{field-upper-limit-value}

In this field, the upper (maximum) limit value for the schedule type should be entered. If this field is left blank, then the schedule type is not limited to a minimum/maximum value range.

\paragraph{Field: Numeric Type}\label{field-numeric-type}

This field designates how the range values are validated. Using \textbf{Continuous} in this field allows for all numbers, including fractional amounts, within the range to be valid. Using \textbf{Discrete} in this field allows only integer values between the minimum and maximum range values to be valid.

\paragraph{Field: Unit Type}\label{field-unit-type}

This field is used to indicate the kind of units that may be associated with the schedule that references the ScheduleTypeLimits object. It is used by IDF Editor to display the appropriate SI and IP units. This field is not used by EnergyPlus. The available options are shown below. If none of these options are appropriate, select \textbf{Dimensionless.}

\begin{itemize}
\item
  Dimensionless
\item
  Temperature
\item
  DeltaTemperature
\item
  PrecipitationRate
\item
  Angle
\item
  Convection Coefficient
\item
  Activity Level
\item
  Velocity
\item
  Capacity
\item
  Power
\item
  Availability
\item
  Percent
\item
  Control
\item
  Mode
\end{itemize}

Several IDF Examples will illustrate the use:

\begin{lstlisting}
ScheduleTypeLimits,Any Number;  ! Not limited
ScheduleTypeLimits,Fraction, 0.0 , 1.0 ,CONTINUOUS;
ScheduleTypeLimits,Temperature,-60,200,CONTINUOUS;
ScheduleTypeLimits,Control Type,0,4,DISCRETE;
ScheduleTypeLimits,On/Off,0,1,DISCRETE;
\end{lstlisting}

\subsection{Day Schedules}\label{day-schedules}

The day schedules perform the assignment of pieces of information across a 24 hour day. This can occur in various fashions including a 1-per hour assignment, a user specified interval scheme or a list of values that represent an hour or portion of an hour.

\subsection{Schedule:Day:Hourly}\label{scheduledayhourly}

The Schedule:Day:Hourly contains an hour-by-hour profile for a single simulation day.

\subsubsection{Inputs}\label{inputs-1-039}

\paragraph{Field: Name}\label{field-name-1-038}

This field should contain a unique (within all DaySchedules) designation for this schedule. It is referenced by WeekSchedules to define the appropriate schedule values.

\paragraph{Field: Schedule Type Limits Name}\label{field-schedule-type-limits-name-000}

This field contains a reference to the Schedule Type Limits object. If found in a list of Schedule Type Limits (see \hyperref[scheduletypelimits]{ScheduleTypeLimits} object above), then the restrictions from the referenced object will be used to validate the hourly field values (below).

\paragraph{Field: Hour Values (1-24)}\label{field-hour-values-1-24}

These fields contain the hourly values for each of the 24 hours in a day. (Hour field 1 represents clock time 00:00:01 AM to 1:00:00 AM, hour field 2 is 1:00:01 AM to 2:00:00 AM, etc.) The values in these fields will be passed to the simulation as indicated for ``scheduled'' items.

An IDF example:

\begin{lstlisting}
Schedule:Day:Hourly, Day On Peak, Fraction,
  0.,0.,0.,0.,0.,0.,0.,0.,0.,1.,1.,1.,1.,1.,1.,1.,1.,1.,0.,0.,0.,0.,0.,0.;
\end{lstlisting}

\subsection{Schedule:Day:Interval}\label{scheduledayinterval}

The Schedule:Day:Interval introduces a slightly different way of entering the schedule values for a day. Using the intervals, you can shorten the ``hourly'' input of the ``\hyperref[scheduledayhourly]{Schedule:Day:Hourly}'' object to 2 fields. And, more importantly, you can enter an interval that represents only a portion of an hour. Schedule values are ``given'' to the simulation at the zone timestep, so there is also a possibility of ``interpolation'' from the entries used in this object to the value used in the simulation.

\subsubsection{Inputs}\label{inputs-2-036}

\paragraph{Field: Name}\label{field-name-2-034}

This field should contain a unique (within all DaySchedules) designation for this schedule. It is referenced by WeekSchedules to define the appropriate schedule values.

\paragraph{Field: Schedule Type Limits Name}\label{field-schedule-type-limits-name-1-000}

This field contains a reference to the Schedule Type Limits object. If found in a list of Schedule Type Limits (see \hyperref[scheduletypelimits]{ScheduleTypeLimits} object above), then the restrictions from the referenced object will be used to validate the hourly field values (below).

\paragraph{Field: Interpolate to Timestep}\label{field-interpolate-to-timestep}

Three possible inputs are available for this field: Average, Linear, and No. The default value is No.

If ``Average'' is entered, it is used to apply values that aren't coincident with the given timestep (ref: Timestep) intervals. If ``Average'' is entered, then any intervals entered here will be interpolated/averaged and that value will be used at the appropriate minute in the hour. For example, if ``Average'' is entered and the interval is every 15 minutes (say a value of 0 for the first 15 minutes, then 0.5 for the second 15 minutes) AND there is a 10 minute timestep for the simulation: the value at 10 minutes will be 0 and the value at 20 minutes will be 0.25. In earlier versions of EnergyPlus, this was the Yes option for the field.

If ``No'' is entered, then the value that occurs on the appropriate minute in the hour will be used as the schedule value. For the same input entries but ``no'' for this field, the value at 10 minutes will be 0 and the value at 20 minutes will be 0.5. No is the default for this field.

If ``Linear'' is entered, then the value that is used is based on the linear interpolation between successive values. With linear, if the value at 1:00 is 0.0 and the value at 2:00 is 10.0, with fifteen minute timesteps, the value at 1:15 would be 2.5, the value at 1:30 would be 5.0 and the value at 1:45 would be 7.5.

\paragraph{Field-Set: Time and Value (extensible object)}\label{field-set-time-and-value-extensible-object}

To specify each interval, both an ``until'' time (which includes the designated time) and the value must be given.  This object is extensible, so additional pairs of the following two fields can be added to the end of this object.

\paragraph{Field: Time}\label{field-time}

The value of each field should represent clock time (standard) in the format ``Until: HH:MM''. 24 hour clock format (i.e.~1PM is 13:00) is used. Note that Until: 7:00 includes all times up through 07:00 or 7am.

\paragraph{Field: Value}\label{field-value}

This represents the actual value to be passed to the simulation at the appropriate timestep. (Using interpolation value as shown above). Limits on the values are indicated by the Schedule Type Limits Name field of this object.

And an example of use:

\begin{lstlisting}
Schedule:Day:Interval,
  dd winter rel humidity, !- Name
  Percent,                !- Schedule Type Limits Name
  No,                     !- Interpolate to Timestep
  until: 24:00,           !- Time 1
  74;                     !- Value Until Time 1
\end{lstlisting}

\subsection{Schedule:Day:List}\label{scheduledaylist}

To facilitate possible matches to externally generated data intervals, this object has been included. In similar fashion to the \hyperref[scheduledayinterval]{Schedule:Day:Interval} object, this object can also include ``sub-hourly'' values but must represent a complete day in its list of values.

\subsubsection{Inputs}\label{inputs-3-032}

\paragraph{Field: Name}\label{field-name-3-028}

This field should contain a unique (within all day schedules) designation for this schedule. It is referenced by week schedules to define the appropriate schedule values.

\paragraph{Field: Schedule Type Limits Name}\label{field-schedule-type-limits-name-2-000}

This field contains a reference to the Schedule Type Limits object. If found in a list of Schedule Type Limits (see \hyperref[scheduletypelimits]{ScheduleTypeLimits} object above), then the restrictions from the referenced object will be used to validate the hourly field values (below).

\paragraph{Field: Interpolate to Timestep}\label{field-interpolate-to-timestep-1}

Three possible inputs are available for this field: Average, Linear, and No. The default value is No.

If ``Average'' is entered, it is used to apply values that aren't coincident with the given timestep (ref: Timestep) intervals. If ``Average'' is entered, then any intervals entered here will be interpolated/averaged and that value will be used at the appropriate minute in the hour. For example, if ``Average'' is entered and the interval is every 15 minutes (say a value of 0 for the first 15 minutes, then 0.5 for the second 15 minutes) AND there is a 10 minute timestep for the simulation: the value at 10 minutes will be 0 and the value at 20 minutes will be 0.25. In earlier versions of EnergyPlus, this was the Yes option for the field.

If ``No'' is entered, then the value that occurs on the appropriate minute in the hour will be used as the schedule value. For the same input entries but ``no'' for this field, the value at 10 minutes will be 0 and the value at 20 minutes will be 0.5. No is the default for this field.

If ``Linear'' is entered, then the value that is used is based on the linear interpolation between successive values. With linear, if the value at 1:00 is 0.0 and the value at 2:00 is 10.0, with fifteen minute timesteps, the value at 1:15 would be 2.5, the value at 1:30 would be 5.0 and the value at 1:45 would be 7.5.

\paragraph{Field: Minutes Per Item}\label{field-minutes-per-item}

This field allows the ``list'' interval to be specified in the number of minutes for each item. The value here must be \textless{}= 60 and evenly divisible into 60 (same as the timestep limits).

\paragraph{Field Value 1 (same definition for each value -- up to 1440 (24*60) allowed)}\label{field-value-1-same-definition-for-each-value-up-to-1440-2460-allowed}

This is the value to be used for the specified number of minutes.

For example:

\begin{lstlisting}
Schedule:Day:List,
  Myschedule, ! name
  Fraction,   ! Schedule type
  No,         ! Interpolate value
  30,         ! Minutes per item
  0.0,        ! from 00:01 to 00:30
  0.5,        ! from 00:31 to 01:00
  <snipped>
\end{lstlisting}

\subsection{Week Schedule(s)}\label{week-schedules}

The week schedule object(s) perform the task of assigning the day schedule to day types in the simulation. The basic week schedule is shown next.

\subsection{Schedule:Week:Daily}\label{scheduleweekdaily}

\subsubsection{Inputs}\label{inputs-4-029}

\paragraph{Field: Name}\label{field-name-4-025}

This field should contain a unique (within all WeekSchedules) designation for this schedule. It is referenced by Schedules to define the appropriate schedule values.

\paragraph{Field: Schedule Day Name Fields (12 day types -- Sunday, Monday, \ldots{} )}\label{field-schedule-day-name-fields-12-day-types-sunday-monday}

These fields contain day schedule names for the appropriate day types. Days of the week (or special days as described earlier) will then use the indicated hourly profile as the actual schedule value.

An IDF example:

\begin{lstlisting}
Schedule:Week:Daily, Week on Peak,
  Day On Peak,Day On Peak,Day On Peak,
  Day On Peak,Day On Peak,Day On Peak,
  Day On Peak,Day On Peak,Day On Peak,
  Day On Peak,Day On Peak,Day On Peak;
\end{lstlisting}

\subsection{Schedule:Week:Compact}\label{scheduleweekcompact}

Further flexibility can be realized by using the Schedule:Week:Compact object. In this the fields, after the name is given, a ``for'' field is given for the days to be assigned and then a dayschedule name is used.

\subsubsection{Inputs}\label{inputs-5-026}

\paragraph{Field:Name}\label{fieldname}

This field should contain a unique (within all WeekSchedules) designation for this schedule. It is referenced by Schedules to define the appropriate schedule values.

\paragraph{Field-Set -- DayType List\#, Schedule:Day Name \#}\label{field-set-daytype-list-scheduleday-name}

Each assignment is made in a pair-wise fashion. First the ``days'' assignment and then the dayschedule name to be assigned. The entire set of day types must be assigned or an error will result.

\paragraph{Field: DayType List \#}\label{field-daytype-list}

This field can optionally contain the prefix ``For'' for clarity. Multiple choices may then be combined on the line. Choices are: Weekdays, Weekends, Holidays, Alldays, SummerDesignDay, WinterDesignDay, Sunday, Monday, Tuesday, Wednesday, Thursday, Friday, Saturday, CustomDay1, CustomDay2. In fields after the first ``for'', AllOtherDays may also be used. Note that the colon (:) after the For is optional but is suggested for readability.

\paragraph{Field: Schedule:Day Name \#}\label{field-scheduleday-name}

This field contains the name of the day schedule (any of the Schedule:Day object names) that is to be applied for the days referenced in the prior field.

Some IDF examples:

\begin{lstlisting}
Schedule:Week:Compact, Week on Peak,
  For: AllDays,
  Day On Peak;

Schedule:Week:Compact, WeekDays on Peak,
  WeekDays,
  Day On Peak,
  AllOtherDays
  Day Off Peak;
\end{lstlisting}

\subsection{Schedule:Year}\label{scheduleyear}

The yearly schedule is used to cover the entire year using references to week schedules (which in turn reference day schedules). If the entered schedule does not cover the entire year, a fatal error will result.

\subsubsection{Inputs}\label{inputs-6-023}

\paragraph{Field: Name}\label{field-name-5-021}

This field should contain a unique (between Schedule:Year, \hyperref[schedulecompact]{Schedule:Compact}, and \hyperref[schedulefile]{Schedule:File}) designation for the schedule. It is referenced by various ``scheduled'' items (e.g.~\hyperref[lights-000]{Lights}, \hyperref[people]{People}, Infiltration) to define the appropriate schedule values.

\paragraph{Field: Schedule Type Limits Name}\label{field-schedule-type-limits-name-3}

This field contains a reference to the Schedule Type Limits object. If found in a list of Schedule Type Limits (see \hyperref[scheduletypelimits]{ScheduleTypeLimits} object above), then the restrictions from the referenced object will be used to validate the hourly field values (below).

\paragraph{Field Set (WeekSchedule, Start Month and Day, End Month and Day)}\label{field-set-weekschedule-start-month-and-day-end-month-and-day}

Each of the designated fields is used to fully define the schedule values for the indicated time period). Up to 53 sets can be used. An error will be noted and EnergyPlus will be terminated if an incomplete set is entered. Missing time periods will also be noted as warning errors; for these time periods a zero (0.0) value will be returned when a schedule value is requested. Each of the sets has the following 5 fields:

\paragraph{Field: Schedule Week Name \#}\label{field-schedule-week-name}

This field contains the appropriate WeekSchedule name for the designated time period.

\paragraph{Field: Start Month \#}\label{field-start-month}

This numeric field is the starting month for the schedule time period.

\paragraph{Field: Start Day \#}\label{field-start-day}

This numeric field is the starting day for the schedule time period.

\paragraph{Field: End Month \#}\label{field-end-month-000}

This numeric field is the ending month for the schedule time period.

\paragraph{Field: End Day \#}\label{field-end-day}

This numeric field is the ending day for the schedule time period.

Note that there are many possible periods to be described. An IDF example with a single period:

\begin{lstlisting}
Schedule:Year, On Peak, Fraction,
  Week On Peak, 1,1, 12,31;
\end{lstlisting}

And a multiple period illustration:

\begin{lstlisting}
Schedule:Year,CoolingCoilAvailSched,Fraction,
  FanAndCoilAllOffWeekSched, 1,1, 3,31,
  FanAndCoilSummerWeekSched, 4,1, 9,30,
  FanAndCoilAllOffWeekSched, 10,1, 12,31;
\end{lstlisting}

The following definition will generate an error (if any scheduled items are used in the simulation):

\begin{lstlisting}
Schedule:Year,MySchedule,Fraction,4,1,9,30;
\end{lstlisting}

\subsection{Schedule:Compact}\label{schedulecompact}

For flexibility, a schedule can be entered in ``one fell swoop''. Using the Schedule:Compact object, all the features of the schedule components are accessed in a single command. Like the ``regular'' schedule object, each schedule:compact entry must cover all the days for a year. Additionally, the validations for DaySchedule (i.e.~must have values for all 24 hours) and WeekSchedule (i.e.~must have values for all day types) will apply. Schedule values are ``given'' to the simulation at the zone timestep, so there is also a possibility of ``interpolation'' from the entries used in this object to the value used in the simulation.

This object is an unusual object for description. For the data the number of fields and position are not set, they cannot really be described in the usual Field \# manner. Thus, the following description will list the fields and order in which they must be used in the object. The name and schedule type are the exceptions:

\subsubsection{Inputs}\label{inputs-7-023}

\paragraph{Field: Name}\label{field-name-6-019}

This field should contain a unique (between \hyperref[scheduleyear]{Schedule:Year}, Schedule:Compact, and \hyperref[schedulefile]{Schedule:File}) designation for the schedule. It is referenced by various ``scheduled'' items (e.g.~\hyperref[lights-000]{Lights}, \hyperref[people]{People}, Infiltration) to define the appropriate schedule values.

\paragraph{Field: Schedule Type Limits Name}\label{field-schedule-type-limits-name-4}

This field contains a reference to the Schedule Type Limits object. If found in a list of Schedule Type Limits (see \hyperref[scheduletypelimits]{ScheduleTypeLimits} object above), then the restrictions from the referenced object will be used to validate the hourly field values (below).

\paragraph{Field-Set (Through, For, Interpolate, Until, Value)}\label{field-set-through-for-interpolate-until-value}

Each compact schedule must contain the elements Through (date), For (days), Interpolate (optional), Until (time of day) and Value. In general, each of the ``titled'' fields must include the ``title''. Note that the colon (:) after these elements (Through, For, Until) is optional but is suggested for readability.

\paragraph{Field: Through}\label{field-through}

This field starts with ``Through:'' and contains the ending date for the schedule period (may be more than one). Refer to Table~\ref{table:date-field-interpretation}. Date Field Interpretation for information on date entry -- note that only Month-Day combinations are allowed for this field. Each ``through'' field generates a new WeekSchedule named ``Schedule Name''\_wk\_\# where \# is the sequential number for this compact schedule.

\paragraph{Field: For}\label{field-for}

This field starts with ``For:'' and contains the applicable days (reference the compact week schedule object above for complete description) for the 24 hour period that must be described. Each ``for'' field generates a new DaySchedule named ``Schedule Name''\_dy\_\# where \# is the sequential number for this compact schedule.

\paragraph{Field: Interpolate (optional)}\label{field-interpolate-optional}

This field, if used, starts with ``Interpolate:'' and contains the word ``Average'', ``Linear'' or ``No''. If this field is not used, it should not be blank -- rather just have the following field appear in this slot.

If ``Average'' is entered, it is used to apply values that aren't coincident with the given timestep (ref: Timestep) intervals. If ``Average'' is entered, then any intervals entered here will be interpolated/averaged and that value will be used at the appropriate minute in the hour. For example, if ``Average'' is entered and the interval is every 15 minutes (say a value of 0 for the first 15 minutes, then .5 for the second 15 minutes) AND there is a 10 minute timestep for the simulation: the value at 10 minutes will be 0 and the value at 20 minutes will be .25. In earlier versions of EnergyPlus, this was the Yes option for the field.

If ``No'' is entered, then the value that occurs on the appropriate minute in the hour will be used as the schedule value. For the same input entries but ``no'' for this field, the value at 10 minutes will be 0 and the value at 20 minutes will be .5. No is the default for this field.

If ``Linear'' is entered, then the value that is used is based on the linear interpolation between successive values. With linear, if the value at 1:00 is 0.0 and the value at 2:00 is 10.0, with fifteen minute timesteps, the value at 1:15 would be 2.5, the value at 1:30 would be 5.0 and the value at 1:45 would be 7.5.

\paragraph{Field: Until}\label{field-until}

This field contains the ending time (again, reference the interval day schedule discussion above) for the current days and day schedule being defined.

\paragraph{Field: Value}\label{field-value-1}

Finally, the value field is the schedule value for the specified time interval.

And, some IDF examples:

\begin{lstlisting}
Schedule:Compact,
  POFF,       !- Name
  Fraction,   !- Schedule Type Limits Name
  Through: 4/30,
  For: AllDays,
  Until: 24:00, 1.0,
  Through: 12/31,
  For: Weekdays,
    Until: 7:00,   .1,
    Until: 17:00, 1.0,
    Until: 24:00,  .1,
  For: Weekends Holidays,
    Until: 24:00,  .1,
  For: AllOtherDays,
    Until: 24:00,  .1;

! Schedule Continuous
Schedule:Compact,
  Continuous,
  on/off,
  Through: 12/31,
  For: AllDays,
    Until: 24:00, 1.0;

! Schedule Daytime Ventilation
Schedule:Compact,
  Daytime Ventilation,
  Fraction,
  Through: 12/31,
  For: Weekdays SummerDesignDay,
    Until: 08:00, 0.0,
    Until: 18:00, 1.0,
    Until: 24:00, 0.0,
  For: Weekends WinterDesignDay,
    Until: 10:00, 0.0,
    Until: 16:00, 1.0,
    Until: 24:00, 0.0,
  For: Holidays AllOtherDays,
    Until: 24:00, 0.0;
\end{lstlisting}

\subsection{Schedule:Constant}\label{scheduleconstant}

The constant schedule is used to assign a constant hourly value. This schedule is created when a fixed hourly value is desired to represent a period of interest (e.g., always on operation mode for supply air fan).

\subsubsection{Inputs}\label{inputs-8-021}

\paragraph{Field: Name}\label{field-name-7-018}

This field should contain a unique name designation for this schedule. It is referenced by Schedules to define the appropriate schedule value.

\paragraph{Field: Schedule Type Limits Name}\label{field-schedule-type-limits-name-5}

This field contains a reference to the Schedule Type Limits object. If found in a list of Schedule Type Limits (see \hyperref[scheduletypelimits]{ScheduleTypeLimits} object above), then the restrictions from the referenced object will be used to validate the hourly field values (below).

\paragraph{Field: Hourly Value}\label{field-hourly-value}

This field contains a constant real value. A fixed value is assigned as an hourly value.

An IDF example:

\begin{lstlisting}
Schedule:Constant,
  AlwaysOn,     !- Name
  On/Off,       !- Schedule Type Limits Name
  1.0;          !- Hourly Value

ScheduleTypeLimits,
  On/Off,       !- Name
  0,            !- Lower Limit Value
  1,            !- Upper Limit Value
  DISCRETE,     !- Numeric Type
  Availability; !- Unit Type
\end{lstlisting}

\subsection{Schedule:File}\label{schedulefile}

At times, data is available from a building being monitored or for factors that change throughout the year. The Schedule:File object allows this type of data to be used in EnergyPlus as a schedule. Schedule:File can also be used to read in hourly or sub-hourly schedules computed by other software or developed in a spreadsheet or other utility.

The format for the data file referenced is a text file with values separated by commas (or other optional delimiters) with one line per hour. The file may contain header lines that are skipped. The file should contain values for an entire year (8760 or 8784 hours of data) or a warning message will be issued. Multiple schedules may be created using a single external data file or multiple external data files may be used. The first row of data must be for January 1, hour 1 (or timestep 1 for subhourly files).

Schedule:File may be used along with the \hyperref[fuelfactors]{FuelFactors} object and TDV files in the DataSets directory to compute Time Dependent Valuation based source energy as used by the California Energy Commission's Title 24 Energy Code. See Fuel Factor for more discussion on Time Dependent Valuation.

Two optional fields: \textbf{Interpolate to Timestep} and \textbf{Minutes per Item} allow for the input of sub-hourly schedules (similar to the \hyperref[scheduledaylist]{Schedule:Day:List} object).

\subsubsection{Inputs}\label{inputs-9-019}

\paragraph{Field: Name}\label{field-name-8-018}

This field should contain a unique (between \hyperref[scheduleyear]{Schedule:Year}, \hyperref[schedulecompact]{Schedule:Compact}, and Schedule:File) designation for the schedule. It is referenced by various ``scheduled'' items (e.g.~\hyperref[lights-000]{Lights}, \hyperref[people]{People}, Infiltration, \hyperref[fuelfactors]{FuelFactors}) to define the appropriate schedule values.

\paragraph{Field: Schedule Type Limits Name}\label{field-schedule-type-limits-name-6}

This field contains a reference to the Schedule Type Limits object. If found in a list of Schedule Type Limits (see \hyperref[scheduletypelimits]{ScheduleTypeLimits} object above), then the restrictions from the referenced object will be used to validate the hourly field values (below).

\paragraph{Field: File Name}\label{schedulefile-field-file-name}

This field contains the name of the file that contains the data for the schedule. The field should include a full path with file name, for best results. The field must be \(\le\) 100 characters. The file name must not include commas or an exclamation point. A relative path or a simple file name should work with version 7.0 or later when using EP-Launch even though EP-Launch uses temporary directories as part of the execution of EnergyPlus. If using RunEPlus.bat to run EnergyPlus from the command line, a relative path or a simple file name may work if RunEPlus.bat is run from the folder that contains EnergyPlus.exe.

\paragraph{Field: Column Number}\label{field-column-number}

The column that contains the value to be used in the schedule. The first column is column one. If no data for a row appears for a referenced column the value of zero is used for the schedule value for that hour.

\paragraph{Field: Rows to Skip at Top}\label{field-rows-to-skip-at-top}

Many times the data in a file contains rows (lines) that are describing the files or contain the names of each column. These rows need to be skipped and the number of skipped rows should be entered for this field. The next row after the skipped rows must contain data for January 1, hour 1.

\paragraph{Field: Number of Hours of Data}\label{field-number-of-hours-of-data}

The value entered in this field should be either 8760 or 8784 as the number of hours of data. 8760 does not include the extra 24 hours for a leap year (if needed). 8784 will include the possibility of leap year data which can be processed according to leap year indicators in the weather file or specified elsewhere. Note if the simulation does not have a leap year specified but the schedule file contains 8784 hours of data, the first 8760 hours of data will be used. The schedule manager will not know to skip the 24 hours representing February 29.

\paragraph{Field: Column Separator}\label{field-column-separator-001}

This field specifies the character used to separate columns of data if the file has more than one column. The choices are: Comma, Tab, Fixed (spaces fill each column to a fixed width), or Semicolon. The default is Comma.

\paragraph{Field: Interpolate to Timestep}\label{field-interpolate-to-timestep-2}

The value contained in this field directs how to apply values that aren't coincident with the given timestep (ref: TimeStep object) intervals. If ``Yes'' is entered, then any intervals entered here will be interpolated/averaged and that value will be used at the appropriate minute in the hour. If ``No'' is entered, then the value that occurs on the appropriate minute in the hour will be used as the schedule value.

For example, if ``yes'' is entered and the minutes per item is 15 minutes (say a value of 0 for the first 15 minutes, then 0.5 for the second 15 minutes) AND there is a 10 minute timestep for the simulation: the value at 10 minutes will be 0 and the value at 20 minutes will be 0.25. For the same input entries but ``no'' for this field, the value at 10 minutes will be 0 and the value at 20 minutes will be 0.5.

\paragraph{Field: Minutes Per Item}\label{field-minutes-per-item-1}

This field represents the number of minutes for each item -- in this case each line of the file. The value here must be \(\le\) 60 and evenly divisible into 60 (same as the timestep limits).

Here is an IDF example:

\begin{lstlisting}
Schedule:File,
  elecTDVfromCZ01res, !- Name
  Any Number,         !- ScheduleType
  TDV_kBtu_CTZ01.csv, !- Name of File
  2,                  !- Column Number
  4,                  !- Rows to Skip at Top
  8760,               !- Number of Hours of Data
  Comma;              !- Column Separator
\end{lstlisting}

or with a relational path:

\begin{lstlisting}
Schedule:File,
  elecTDVfromCZ01res, !- Name
  Any Number,         !- ScheduleType
  DataSets\TDV\TDV_kBtu_CTZ01.csv,  !- Name of File
  2,                  !- Column Number
  4;                  !- Rows to Skip at Top
\end{lstlisting}

A sub-hourly indication. Note that this is identical to an hourly file because there are 60 minutes per item -- the number of hours defaults to 8760 and the column separator defaults to a comma. If the number of minutes per item had been, say, 15, then the file would need to contain 8760*4 or 35,040 rows for this item.

\begin{lstlisting}
Schedule:File,
  elecTDVfromCZ06com, !- Name
  Any Number,         !- Schedule Type Limits Name
  DataSets\TDV\TDV_2008_kBtu_CTZ06.csv, !- File Name
  1,                  !- Column Number
  4,                  !- Rows to Skip at Top
  ,                   !- Number of Hours of Data
  ,                   !- Column Separator
  ,                   !- Interpolate to Timestep
  60;                 !- Minutes per Item
\end{lstlisting}

\paragraph{Field: Adjust Schedule for Daylight Savings}\label{adjust-schedule-for-daylight-savings-1}

This field indicates whether or not to shift the schedule forward an hour during daylight savings time. If ``Yes'' the schedule skips forward an hour in the CSV on the ``spring forward'' day and repeats an hour on the ``fall back'' day. If  ``No'', the schedule is read and applied in standard time, without adjusting for daylight savings time (\textbf{\hyperref[runperiodcontroldaylightsavingtime]{RunPeriodControl:DaylightSavingTime}}). This indicator is defaulted to ``Yes'' if no input is provided. 

Here is an IDF example:

\begin{lstlisting}
	Schedule:File,
	TestName,	!- Name
	Any Number,	!- Schedule Type Limits Name
	schedulefile.csv, !- File Name
	2,		!- Column Number
	1,		!- Rows to Skip at Top
	8760,		!- Number of Hours of Data
	Comma,		!- Column Separator
	No,		!- Interpolate to Timestep
	60,		!- Minutes per Item
	No;		!- Adjust Schedule for Daylight Savings
\end{lstlisting}

\subsubsection{Outputs}\label{outputs-031}

An optional report can be used to gain the values described in the previous Schedule objects. This is a condensed reporting that illustrates the full range of schedule values---in the style of input: DaySchedule, WeekSchedule, Annual Schedule.

\begin{lstlisting}
! will give them on hourly increments (day schedule resolution)
Output:Schedules, Hourly;
! will give them at the timestep of the simulation
Output:Schedules, Timestep;
\end{lstlisting}

This report is placed on the eplusout.eio file. Details of this reporting are shown in the Output Details and Examples document.

\paragraph{Schedule Value Output}\label{schedule-value-output}

\begin{lstlisting}
Zone,Average,Schedule Value {[]}
\end{lstlisting}

\paragraph{Schedule Value}\label{schedule-value}

This is the schedule value (as given to whatever entity that uses it). It has no units in this context because values may be many different units (i.e.~temperatures, fractions, watts). For best results, you may want to apply the schedule name when you use this output variable to avoid output proliferation. For example, the following reporting should yield the values shown above, depending on day of week and day type:

\begin{lstlisting}
Output:Variable,People_Shopping_Sch,Schedule Value,hourly;
Output:Variable,Activity_Shopping_Sch,Schedule Value,hourly;
\end{lstlisting}

A wild card $*$ would also work in place of the schedule name. For example, the following syntax would report all the schedule values for all the schedules in an IDF file:
\begin{lstlisting}
Output:Variable,*,Schedule Value,hourly;
\end{lstlisting}

\subsection{Schedule:File:Shading}\label{schedulefileshading}

The \textbf{Schedule:File:Shading} object allows shading schedules to be imported altogether from a file exported using \hyperref[shadowcalculation]{ShadowCalculation} Output External Shading Calculation Results. The object can also be used to read in hourly or sub-hourly schedules of the sunlit fraction of all exterior surfaces computed by other software or developed in a spreadsheet or other utility.

The format for the data file is Comma-separated values (CSV). The CSV file referenced is a text file with values separated by commas (or other optional delimiters) with one line per timestep. Each column stores the annual shading sunlit fraction schedule data of an exterior surface. The sunlit fraction is only overwritten and used when the sun is above horizon at a certain time step. To map to the surface, each column should name its column header exactly the same with the surface name defined in the \textbf{Surface:Detailed} object. If any surface name is missing, the sunlit fraction of this surface will be set to 1.0 during the run period, which means no shading will be assigned for this surface, and a warning message will be issued. The file should contain values for an entire year with the exact same number of rows as the total number of time steps in a year (number of days in a year $\times$ number of hours per day $\times$ number of time steps per hour). Time step should be consistent with the setting in the \textbf{Timestep} object. Otherwise, an error will be issued.

The first row of the CSV file should be the header, and the first column of the CSV file should be the timestamp and is not imported.

With a \textbf{Schedule:File:Shading} object defined, the shading schedules for all exterior surfaces (if defined) can be imported altogether without repeatedly defining \hyperref[schedulefile]{Schedule:File} objects.

\subsubsection{Inputs}\label{inputs-schedule-file-shading}

\paragraph{Field: File Name}\label{schedulefileshading-field-file-name}

This field contains the name of the file that contains the data for the shading schedules. The field should include a full path with file name, for best results. The field must be \(\le\) 100 characters. The file name must not include commas or an exclamation point. A relative path or a simple file name should work with version 7.0 or later when using EP-Launch even though EP-Launch uses temporary directories as part of the execution of EnergyPlus. If using RunEPlus.bat to run EnergyPlus from the command line, a relative path or a simple file name may work if RunEPlus.bat is run from the folder that contains EnergyPlus.exe.

Here is an IDF example:

\begin{lstlisting}
Schedule:File,
  eplusshading.csv;   !- Name of File
\end{lstlisting}
