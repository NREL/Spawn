\section{What is a module anyway?}\label{what-is-a-module-anyway}

\subsection{Program Modules}\label{program-modules}

A module is a Fortran 90/95 programming construct that can be used in various ways. In EnergyPlus, its primary use is to segment a rather large program into smaller, more manageable pieces. Each module is a separate package of source code stored on a separate file. The entire collection of modules, when compiled and linked, forms the executable code of EnergyPlus.

Each module contains source code for closely related data structures and procedures. For instance, the WeatherManager module contains all the weather handling routines in EnergyPlus. The module is contained in the file WeatherManager.f90.~ Another example is PlantPumps. This module contains all the code to simulate pumps in EnergyPlus. It is contained in file PlantPumps.f90.

Of course dividing a program into modules can be done in various ways. We have attempted to create modules that are as self-contained as possible. The philosophy that has been used in creating EnergyPlus is contained in the \href{file:///E:/Docs4PDFs/ProgrammingStandards.pdf}{Programming Standard} reference document.~ Logically, the modules in EnergyPlus form an inverted tree structure. At the top is EnergyPlus. Just below that are ProcessInput and ManageSimulation. At the bottom are the modules such as HVACDamperComponent that model the actual HVAC components.

\subsection{Data Only Modules}\label{data-only-modules}

EnergyPlus also uses modules that primary contain data and data structures that may be used by several modules. These modules form one of the primary ways data is structured and shared in EnergyPlus. An example is the DataEnvironment module. Many parts of the program need access to the outdoor conditions. All of that data is encapsulated in DataEnvironment. Modules that need this data obtain access through a Fortran USE statement. Without such access, modules cannot use or change this data.

Sometimes data modules are extended to perform certain utilities or even getting input for the data structures that are the primary focus of the module but this is not a standard approach.
