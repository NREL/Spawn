\section{Introduction}\label{introduction}

The Input Macros feature increases the flexibility of the EnergyPlus input files. This feature is intended for advanced users who are already familiar with EnergyPlus IDF files and need to prepare input manually. The basic capabilities are:

\begin{itemize}
\item
  Incorporating external files containing pieces of IDF into the main EnergyPlus input stream.
\item
  Selectively accepting or skipping portions of the input.
\item
  Defining a block of input with parameters and later referencing this block.
\item
  Performing arithmetic and logical operations on the input.
\item
  Input macro debugging and listing control.
\end{itemize}

These capabilities are invoked in the EP-MACRO program by using macro commands. Macro commands are preceded by \#\# to distinguish them from regular EnergyPlus input commands. After execution by the EP-MACRO processor, macro commands produce regular lines of EnergyPlus input that are shown in the resultant IDF file (\textbf{out.idf}) and, subsequently, in the EnergyPlus echo print (\textbf{audit.out}). Following are descriptions of the macro commands associated with the above capabilities. A detailed example of input macros is given at the end of this section; you should review it before reading the macro command descriptions.
