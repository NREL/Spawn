\section{RunEPlus details}\label{runeplus-details}

A procedure (batch) file is the normal way to run a console application. The \emph{installed} procedure file \textbf{RunEPlus.bat} can be used to execute EnergyPlus and deal with all the file handling and postprocessing. It can accommodate running the EPMacro program if you name your files appropriately. And it can use ExpandObjects to expand the special ``HVACTemplate'' objects into ``normal'' IDF objects.

The ``set'' statements near the beginning of the procedure file can be customized for each local system. Thus ``program\_path'' should be set to the directory path where the program executable resides on your local computer, ``program\_name'' should be set to the name of the EnergyPlus executable file, ``input\_path'' should be set to the directory path containing the input (IDF) file, and so forth. Each of the path environment variables must have ``\textbackslash{}'' as the final character or things won't run correctly. As mentioned before, the batch file is executed by typing:

RunEPlus \textless{}input\_filename\textgreater{} \textless{}weather\_filename\textgreater{}

where \textless{}input\_filename\textgreater{} is the name of the IDF file, without the file extension, and \textless{}weather\_filename\textgreater{} is the name of the weather file, without the file extension.The \textless{}input\_filename\textgreater{} can also be a complete path to the file (without extension) and it will work.

In addition, RunEPlus can be called from a different directory and the temporary files will be created in the directory it is called from. This enables multiple RunEPlus.bat to be used with multiple processors or a multiple-core processor without the temporary files of one set of simulations interfering with another. Each call to RunEPlus.bat should be from different directories.

Instructions appear at the top of the batch file:

\begin{lstlisting}
:Instructions:
:  Complete the following path and program names.
:  path names must have a following \ or errors will happen
:  does not have the capability to run input macro files (yet)
:   %program_path% contains the path to the executable as well as IDD and is
:                  the root directory
:   %program_name% contains the name of the executable (normally EnergyPlus.exe)
:   %input_path%   contains the path to the input file (passed in as first argument)
:   %output_path%  contains the path where the result files should be stored
:   %post_proc%    contains the path to the post processing program (ReadVarsESO)
:   %weather_path% contains the path to the weather files (used with optional argument 2)
:   %pausing%      contains Y if pause should occur between major portions of
:                  batch file (mostly commented out)
:   %maxcol%       contains "250" if limited to 250 columns otherwise contains
:                  "nolimit" if unlimited (used when calling readVarsESO)
\end{lstlisting}

\begin{lstlisting}
 echo = = = = = %0 (Run EnergyPlus) %1 %2 = = = = = Start = = = = =
 set program_path =
 set program_name = EnergyPlus.exe
 set input_path = ExampleFiles\
 set output_path = Test\
 set post_proc = PostProcess\
 set weather_path = WeatherData\
 set pausing = N
 set maxcol = 250
\end{lstlisting}

\begin{lstlisting}
:  This batch file will perform the following steps:
:
:   1. Clean up directory by deleting old working files from prior run
:   2. Clean up target directory
:   3. Copy %1.idf (input) into In.idf
:   4. Copy %2 (weather) into In.epw
:   5. Execute EnergyPlus
:   6. If available Copy %1.rvi (post processor commands) into Eplusout.inp
:   7. Execute ReadVarsESO.exe (the Post Processing Program)
:   8. If available Copy %1.mvi (post processor commands) into test.mvi
:       or create appropriate input to get meter output from eplusout.mtr
:   9. Execute ReadVarsESO.exe (the Post Processing Program) for meter output
:  10. Copy Eplusout.* to %1.*
:  11. Clean up working directory.
\end{lstlisting}
