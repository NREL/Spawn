\section{HVAC Systems}\label{hvac-systems}

\subsection{System Node Setpoints}\label{system-node-setpoints}

A series of actuators is available for all the setpoints that can be placed on system nodes. System nodes are used to define air and plant loops, and a natural application of EMS is to control the setpoints at these nodes. The node actuators are all called ``System Node Setpoint.''~ There are nine control types:

\begin{itemize}
\item
  Temperature Setpoint, (°C)
\item
  Temperature Minimum Setpoint (°C)
\item
  Temperature Maximum Setpoint (°C)
\item
  Humidity Ratio Setpoint (kgWater/kgDryAir)
\item
  Humidity Ratio Minimum Setpoint (kgWater/kgDryAir)
\item
  Humidity Ratio Maximum Setpoint (kgWater/kgDryAir)
\item
  Mass Flow Rate Setpoint (kg/s)
\item
  Mass Flow Rate Minimum Available Setpoint (kg/s)
\item
  Mass Flow Rate Maximum Available Setpoint (kg/s)
\end{itemize}

Using these actuators is natural with an EMS. Typically, the controller would place the setpoint on the outlet node. Then the component's low-level controller should operate to meet the leaving setpoint. Setting the setpoints on nodes should be a common application for the EMS.

Although all nine possible setpoints are available as EMS actuators, it does not follow that EnergyPlus can use all of them. Most components can use only one or two setpoints. If a component cannot control to meet the setpoints on a node, the actuator will do nothing.

\subsection{Zone HVAC Control}\label{zone-hvac-control}

Several actuators relate to HVAC zone controls for temperature, humidity, and comfort.

\begin{itemize}
\item
  Two actuators called ``Zone Temperature Control'' are available with the control types of ``Heating Setpoint'' and ``Cooling Setpoint.''~ These enable you to directly override the zone-by-zone heating and cooling setpoints. The units are in degrees Celsius. The unique identifier is the zone name.
\item
  Two actuators called ``Zone Humidity Control'' are available with the control types ``Relative Humidity Humidifying Setpoint'' and ``Relative Humidity Dehumidifying Setpoint.'' ~These enable you to directly override the zone-by-zone humidity control setpoints. The units are in percent relative humidity. The unique identifier is the zone name.
\item
  Two actuators called ``Zone Comfort Control'' are available with the control types ``Heating Setpoint'' and ``Cooling Setpoint.''~ These enable you to directly override the zone-by-zone comfort control setpoints. The units are predicted mean vote (PMV). The unique identifier is the zone name.
\end{itemize}

\subsection{Plant Supervisory Control}\label{plant-supervisory-control}

Several levels of actuators are available for on/off supervisory control of Plant systems.

\begin{itemize}
\item
  Each plant (and condenser) loop has an actuator called ``Plant Loop Overall'' available with the control type ``On/Off Supervisory.''~ Setting the value of this actuator to 1.0 directs the overall plant to loop to run normally based on other controls. Setting the value of this actuator to 0.0 directs the overall plant loop to shut down regardless of what other controls indicate.
\item
  Each plant (and condenser) loop has actuators called ``Supply Side Half Loop'' and ``Demand Side Half Loop'' that are available with the control type ``On/Off Supervisory.''~ Setting the value of this actuator to 1.0 directs the plant's loop side to run normally based on other controls. Setting the value of this actuator to 0.0 directs the plant's loop side to shut down regardless of what other controls indicate.
\item
  Each plant (and condenser) loop has a series of actuators called ``Supply Side Branch'' and ``Demand Side Branch'' that are available with the control type ``On/Off Supervisory.''~ These are available for each individual branch in a loop. Setting the value of this actuator to 1.0 directs the plant's branch to run normally based on other controls. Setting the value of this actuator to 0.0 directs the plant's branch to shut down regardless of what other controls indicate.
\item
  Each plant (and condenser) loop has a series of actuators called ``Plant Component *'' that are available with the control type ``On/Off Supervisory.''~ These are available for each individual component on a loop. Setting the value of this actuator to 0.0 directs the component's water flow rate to shut down regardless of what other controls indicate. Any other component on the same branch will also be shut down. Setting the value of this actuator to any value <=1.0 sets the current load for this component to the actuator value times the component maximum operating capacity which is the nominal capacity multiplied by the maximum part load ratio.
\end{itemize}

\subsection{Outdoor Air System Node Conditions}\label{outdoor-air-system-node-conditions}

Actuators called ``Outdoor Air System Node'' are available with control types called ``Drybulb Temperature'' and ``Wetbulb Temperature.''~ The units are degrees Celsius. These actuators are available for all system nodes that are listed in either an OutdoorAir:Node or OutdoorAir:NodeList input object. You should probably set both the drybulb and wetbulb temperatures to ensure a full description of the moist air conditions.  In addition, there are also actuators for "Wind Speed" and "Wind Direction" with the units of meters/second and degrees, respectively.

The air system and many component models require you to specify a node as an outdoor air node to obtain values for the outdoor conditions. For example, outdoor air nodes are used at the inlet to an outdoor air mixer or at the inlet of the heat rejection side of a component model. Typically this is the weather data value for outdoor conditions. But local variations in microclimate may shift the local outdoor air temperature to differ slightly from the weather data. (Currently the only local variation model for this effect in EnergyPlus varies the outdoor air conditions as a function of height.)~ If you want to experiment with other models for local variations in outdoor air conditions, this EMS actuator allows you to override the outdoor temperature at a particular system node with any model that can be implemented in an Erl program. For example (although better models for the changes in conditions may need be developed), this actuator could be used to examine the energy impacts of warmer outdoor air temperatures experienced by a rooftop packaged HVAC system sitting on a black roof or the cooler conditions experienced by a unit that is located on the shaded side of a building.~ Another example is to make use of a separate model, outside of EnergyPlus, for some unique type of component (such as a labyrinth or earth-tube) that preconditions outdoor air; the results of that model could be fed into the air system model in EnergyPlus using these actuators.

\subsection{AirLoopHVAC Availability Status}\label{airloophvac-availability-status}

This actuator is available in all models with central, or primary, air systems that are entered with the object ``AirLoopHVAC'' using the control type called ``Availability Status.''~ Various availability managers use the air loop's availability status to override control of the central air system fan. The fan may be scheduled to be unavailable during certain times to shut down the system when it is not needed. However, there may be times when the air system should be started to protect from freezing, for example. This actuator can force an air system to start up or shut down.

The control is actuated by setting values for the availability status. The settings are numeric, but represent discrete states the status can take. The following settings are valid:

\begin{itemize}
\item
  0.0 ( = NoAction). This tells the air system to do whatever it would usually do without any special override status.
\item
  1.0 ( = ForceOff). This overrides the air system to shut down when it would normally want to run.
\item
  2.0 ( = CycleOn). This overrides the air system to start up when it would normally be off.
\item
  3.0 ( = CycleOnZoneFansOnly). This overrides only the zone fans (not the central fans) if they would normally be off.
\end{itemize}

\subsection{Ideal Loads Air System}\label{ideal-loads-air-system}

An actuator called ``Ideal Loads Air System'' is available with control types called ``Air Mass Flow Rate'' (supply air), ``Outdoor Air Mass Flow Rate,'' ``Air Temperature,'' and ``Air Humidity Ratio.'' ~These are available in models that use the ideal loads air system, formerly known as purchased air. The units are kg/s for mass flow rate, C for temperature and kgWater/kgDryAir for humidity ratio. The unique identifier is the user-defined name of the ZoneHVAC:IdealLoadsAirSystem input object.

For Air Temperature and Air Humidity Ratio, the overrides are absolute. They are applied after all other limits have been checked. For mass flow rate, the overrides are not absolute,the internal controls will still apply the capacity and flow rate limits if defined in the input object. The EMS override will be ignored if the ideal loads system is off (the availability schedule value is zero or it has  been forced ``off'' by an availability manager). If both the Air Mass Flow Rate and Outdoor Air Mass Flow Rate are overridden, the Outdoor Air Mass Flow Rate will not be allowed to be greater than the override value for Air Mass Flow Rate.

\subsection{Fan}\label{fan}

Actuators called ``Fan'' are available with the control types ``Fan Air Mass Flow Rate,'' ``Fan Pressure Rise,'' and ``Fan Total Efficiency.''~ These provide direct control over the fan operation in an air system. The EMS program can override the flow rate by using kg/s. It can override the total pressure rise at the fan by using Pascals. And it can override the fan efficiency on a scale from 0.0 to 1.0. The unique identifier is the name of the fan in the Fan input objects.

An actuator is also available for overriding the autosize value for the fan's design air flow rate.~ This actuator is called ``Fan'' and the control type is ``'Fan Autosized Air Flow Rate'' with units in m\(^{3}\)/s. It is only useful from the calling point named AfterComponentInputReadIn.

\subsection{DX Cooling Coils}\label{dx-cooling-coils}

Actuators are available for overriding the autosize rated airflow rate and total cooling capacity of the Coil:Cooling:DX object.~ Actuators called ``Coil:Cooling:DX:SingleSpeed'' are available with control types ``Autosized Rated Air Flow Rate'' (in m3/s), ``Autosized Rated Total Cooling Capacity'' (in W), and ``Autosized Rated Sensible Heat Ratio'' (in W/W).~ These are only useful from the calling point named AfterComponentInputReadIn.

\subsection{DX Thermal Storage Coils}\label{dx-thermal-storage-coils}

There is an actuator that is available for overriding the operating mode of the DX thermal storage coil object.~ The actuator is called ``Coil:Cooling:DX:SingleSpeed:ThermalStorage'' and is available with the control type ``Operating Mode.''  The operating mode has the following states/values:

\begin{itemize}
\item
0 = Off Mode
\item
1 = Cooling Only Mode
\item
2 = Cooling and Charge Mode
\item
3 = Cooling and Discharge Mode
\item
4 = Charge Only Mode
\item
5 = Discharge Only Mode
\end{itemize}

\subsection{Unitary Equipment}\label{unitary-equipment}

Actuators called ``Unitary HVAC'' are available with the control types ``Sensible Load Request'' and ``Moisture Load Request.''~ These control the operation of unitary equipment. Normally these systems operate to meet zone loads, but these actuators allow you to override the controls of unitary systems. The units are in Watts. The unique identifier is the name of the unitary equipment in the input objects.

Actuators are available for overriding the autosize values related to supply air flow rates some unitary HVAC equipment.~ These actuators allow selectively altering the outcome of sizing routines and are used from the calling point named AfterComponentInputReadIn.~ The units are m\(^{3}\)/s.

\begin{itemize}
\item
  An actuator called ``AirLoopHVAC:Unitary:Furnace:HeatOnly'' is available with control type ``Autosized Supply Air Flow Rate.''
\item
  An actuator called ``AirLoopHVAC:UnitaryHeatOnly'' is available with control type ``Autosized Supply Air Flow Rate.''
\item
  Actuators called ``AirLoopHVAC:Unitary:Furnace:HeatCool,''  ``AirLoopHVAC:UnitaryHeatCool,'' and ``UnitarySystem'' are available with control types ``Autosized Supply Air Flow Rate,''~ ``Autosized Supply Air Flow Rate During Cooling Operation,'' ``Autosized Supply Air Flow Rate During Heating Operation,'' and ``Autosized Supply Air Flow Rate During No Heating or Cooling Operation.''
\item
  An actuator called ``AirLoopHVAC:UnitaryHeatPump:AirToAir'' is available with control type ``Autosized Supply Air Flow Rate.''
\item
  An actuator called ``AirLoopHVAC:UnitaryHeatPump:WaterToAir'' is available with control type ``Autosized Supply Air Flow Rate.''
\end{itemize}

An actuator called ``Coil Speed Control'' is available with control type ``Unitary System DX Coil Speed Value''. This actuator is only available for the `AirLoopHVAC:UnitarySystem` or `AirLoopHVAC:UnitaryHeatPump:AirToAir:Multispeed` object, referencing multispeed DX coils.

The EMS override coil speed value, “Unitary System DX Coil Speed Value”, is a continuous number below maximum coil speed level allowed. With any EMS override coil speed value, if the input value is a integer, the speed level is set as the exact speed value input, with the cycling or speed ratio = 1.0. Otherwise, if the floating point part is greater than zero, the speed level is calculated as the closest integer greater than the EMS speed value, and the cycling/speed ratio is set as the floating point part of the EMS speed value. For example, if EMS overrides coil speed value = 1.2, the speed level number is set as 2 with a speed ratio at 0.2.

It should be noted that if there's no cooling or heating or moisture load presented at any time step, the coil will shut off the coil regardless of the EMS speed setting. If the Outside Dry Bulb Temperature is lower than the minimum outdoor operating temperature for heat pump compressor, the coil will also shut off regardless the EMS speed setting.

\subsection{AirTerminal:SingleDuct:ConstantVolume:NoReheat}\label{airTerminalsingleductconstantvolumenoreheat}

An actuator called ``AirTerminal:SingleDuct:ConstantVolume:NoReheat'' is available with a control type called ``Mass Flow Rate.''~ This actuator is available in models that use the single duct constant volume no reheat air terminal. The units are kg/s. This actuator is used to control the mass flow rate. Normally, the flow rate of single duct constant volume no reheat air terminals is fixed by the input, sizing results or OA requirement for DOA system, but this actuator provides a way to override the flow with Erl programs.

\subsection{Outdoor Air Controller}\label{outdoor-air-controller}

An actuator called ``Outdoor Air Controller'' is available with the control type called ``Air Mass Flow Rate.'' ~This provides override control over the rate of outdoor air. The units are kg/s. The unique identifier is the name of the Controller:OutdoorAir input object. The actuated mass flow rate is not allowed to be greater than the current system mixed air flow rate.

\subsection{Plant Equipment Operation}\label{plant-equipment-operation}

An actuator called ``Plant Equipment Operation'' is available with a control type called ``Distributed Load Rate.''~ The units are Watts.  This allows the override of the assigned distributed load for plant equipment.

\subsection{Plant Load Profile}\label{plant-load-profile}

Actuators called ``Plant Load Profile'' are available with the control types called ``Mass Flow Rate'' (in \si{\massFlowRate}) and ``Power'' (in W). The unique identifier is the name of the LoadProfile:Plant input object. These actuators provide override control over the loads placed on a plant system by a plant load profile.

\subsection{Pump}\label{pump}

An actuator called ``Pump'' is available with the control type ``Pump Mass Flow Rate'' (in \si{\massFlowRate}). This allows you to override the flow rate produced by a pump.

Another actuator called ``Pump'' is available with the control type ``Pump Pressure Rise'' (in \si{\pascal}). This allows you to override the pump pressurise rise and therefore the power consumption of the pump, which is calculated as:

\begin{equation}
    P = \frac{\dot{Q} \cdot \Delta P_{override}}{\eta_{total}}
\end{equation}

where $P$ is the pump power (\si{\watt}), $\dot{Q}$ is the volume flow rate (\si{\volumeFlowRate}), $\eta_{total}$ is the pump total efficiency (\%)
and $\Delta P_{override}$ is your EMS-overriden pressure rise (\si{\pascal}).

The unique identifier in both these actuator is the name of Pump Input object.

\subsection{Window Air Conditioner}\label{window-air-conditioner}

An actuator called ``Window Air Conditioner'' is available with a control type called ``Part Load Ratio.''~ This is nondimensional and takes numbers between 0.0 and 1.0. The unique identifier is the name of the ZoneHVAC:WindowAirConditioner input object.

\subsection{Low Temperature Radiant Hydronic}\label{low-temperature-radiant-hydronic}

Actuators called ``Hydronic Low Temp Radiant'' and ``Constant Flow Low Temp Radiant'' are available with the control type ``Water Mass Flow Rate'' (in \si{\massFlowRate}). This allows you to override the flow of water through hydronic radiant systems. The unique identifier is the name of either the ZoneHVAC:LowTemperatureRadiant:VariableFlow or ZoneHVAC:LowTemperature\\
Radiant:ConstantFlow input objects.

\subsection{Variable Refrigerant Flow Heat Pump Air Conditioner}\label{variable-refrigerant-flow-heat-pump-air-conditioner}

An actuator called ``Variable Refrigerant Flow Heat Pump'' is available with a control type called ``Operating Mode.''~ This is nondimensional and takes numbers between 0.0 and 2.0 where 0.0 means the system is off, 1.0 means the system is in cooling mode, and 2.0 means the system is in heating mode. The unique identifier is the name of the AirConditioner:VariableRefrigerantFlow input object.

\subsection{Variable Refrigerant Flow Terminal Unit}\label{variable-refrigerant-flow-terminal-unit}

An actuator called ``Variable Refrigerant Flow Terminal Unit'' is available with a control type called ``Part Load Ratio.''~ This is nondimensional and takes numbers between 0.0 and 1.0. The unique identifier is the name of the ZoneHVAC:TerminalUnit:VariableRefrigerantFlow input object. This control over rides the part-load ratio of the terminal unit and can be applied only when the cooling or heating minimum and maximum outdoor temperature limits of the condenser (i.e., the AirConditioner:VariableRefrigerantFlow object) are not exceeded.
