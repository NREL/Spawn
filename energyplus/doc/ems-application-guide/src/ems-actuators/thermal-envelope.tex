\section{Thermal Envelope}\label{thermal-envelope}

\subsection{Window Shading Control}\label{window-shading-control}

Actuators called ``Window Shading Control'' are available with actuated component control types called ``Control Status'' in [ShadeStatus] and ``Slat Angle'' in [degree]. This actuator is available in models that also have the WindowShadingControl input objects. The user-defined name for the WindowShadingControl is not used to identify unique window shading controls; rather, the window name is used to identify the actuator. This is because there could be multiple windows, all with shades, each of which is governed by a single WindowShadingControl input object. The EMS actuator can override the control actions for each window separately.

The control is actuated by setting values for the control status. The settings are numeric but represent discrete states the control can take. The appropriate values depend on the type and position of the shading device. There are three basic types of shading devices:~ switchable glazings, shades, and blinds. (Shades are described with WindowMaterial:Shade input objects. Blinds are described with WindowMaterial:Blind input objects.)~ Shades and blinds can be situated in the exterior, between the glass, or in the interior.

The following settings are valid and must be used:

\begin{itemize}
\item
  --1.0: ~~~ No shading device.
\item
  0.0: ~~~~~ Shading device is off (applies to shades and blinds).
\item
  1.0: ~~~~~ Interior shade is on.
\item
  2.0:~~~~~~ Glazing is switched to a darker state (switchable glazings only).
\item
  3.0: ~~~~~ Exterior shade is on.
\item
  4.0: ~~~~~ Exterior screen is on.
\item
  6.0: ~~~~~ Interior blind is on.
\item
  7.0: ~~~~~ Exterior blind is on.
\item
  8.0: ~~~~~ Between-glass shade is on.
\item
  9.0: ~~~~~ Between-glass blind is on.
\end{itemize}

\subsection{Slat Angle}\label{slat-angle}

If the shading device is a blind, there is also a control type called ``Slat Angle.''~ This angle control is a continuous numeric value from 0.0 to 180.0. The angle is defined as between the glazing system's outward normal and the slat's outward normal (see the diagram in the input output reference under WindowMaterial:Blind).

\subsection{Surface Convection Heat Transfer Coefficient}\label{surface-convection-heat-transfer-coefficient}

Two actuators called ``Surface'' are available with the control types of ``Interior Surface Convection Heat Transfer Coefficient'' and ``Exterior Surface Convection Heat Transfer Coefficient.''~ These provide direct control over the convection coefficient. The units are W/m\(^{2}\)-K. The unique identifier is the surface name.

This actuator controls the heat transfer modeling. Changes in air distribution systems can affect the interior surface convection coefficients. A sheltered exterior surface may have a lower surface heat transfer coefficient. This actuator provides a method of applying new models for convection coefficients.

\subsection{Material Surface Properties}\label{material-surface-properties}

Three actuators are available for controlling the surface properties material
related to absorptance. Those material layers used in a Construction object that
lie at the outside and the inside of the assembly determine the surface
properties of a heat transfer surface.~ Actuators called ``Material'' are
available with the control types ``Surface Property Solar Absorptance,''
``Surface Property Thermal Absorptance,'' and ``Surface Property Visible
Absorptance.''~ These are dimensionless parameters between 0.0 and 1.0.~ These
actuators are useful for modeling switchable coatings such as thermochromic
paints. Note that for a single-layer construction, both the inside and outside
properties will be overwritten. For overwriting outside properties alone, please
use ``MaterialProperty:VariableAbsorptance'' (see InputOutputReference).

\subsection{Surface Construction State}\label{surface-construction-state}

An actuator is available for controlling the construction assigned to a surface that can be useful for modeling dynamic technologies for thermal envelopes.~ These actuators are called ``Surface'' and have a control type ``Construction State.''~ This actuator is used in conjunction with the input object called EnergyManagementSystem:ConstructionIndexVariable.~ Each Construction object defined in an input file has an index that points to where the data for that construction are stored in the program's internal data structures.~ The EnergyManagementSystem:ConstructionIndexVariable input object is used to create and fill a global Erl variable with the value that points to the specific construction named in the object. The Erl variable is what you assign to the construction state actuator's variable to override the construction assigned to a particular surface.~ When the actuator is set to Null, the surface reverts to the Construction originally referenced by the surface in the input file.

When using surface construction state to change window properties in combination with daylighting calculations, then the Calculation Method in the ShadowCalculation object must be set to TimestepFrequency. This will cause the daylighting factor calculations to be updated every timestep.

Using the surface construction state actuator brings with it a high degree of
risk when it comes to modeling thermal heat capacity and transient heat
transfer in opaque surfaces. If this actuator is used inappropriately, for
example to assign different constructions, to a single surface, that have
radically different heat storage capacities, then the heat transfer modeling
results may not be physically accurate. When a construction state is overridden
using this actuator, the thermal history data that evolved while using the
previous construction are reused for the new construction. When this actuator
is used, the program attempts to detect if incompatible constructions are being
assigned. In some cases it issues a warning and allows the assignments to
proceed, in others it warns and doesn't allow the assignment to proceed. If the
original construction assigned to a surface has internal source/sink (defined
using ConstructionProperty:InternalHeatSource) then any assignments to the
surface must also be internal source constructions. If using the heat transfer
algorithm called ConductionFiniteDifference, then the constructions must have
the same number of finite difference nodes or the assignment is not allowed.
The construction state actuator cannot be used in conjunction with the heat
transfer algorithms called ConductionFiniteDifferenceSimplified or
CombinedHeatAndMoistureFiniteElement.

\subsection{Surface Boundary Conditions}\label{surface-boundary-conditions}

Several actuators, called ``Surface,''~ are available for controlling the temperature and wind speed used for ``Outdoors''~ boundary conditions during heat transfer calculations at the outside face of surfaces. Each heat transfer surface will make available actuators with the following five control types:

\begin{itemize}
\item
  ``Outdoor Air Drybulb Temperature.''~ This is the drybulb temperature of the ambient outdoor air exposed to the surface, in degrees C. This is the temperature used for surface convection heat transfer boundary conditions on the outdoor, outside-face, of the surface exposed to exterior environment (unless it is raining).
\item
  ``Outdoor Air Wetbulb Temperature.''~ This is the wetbulb temperature of the ambient outdoor air exposed to the surface, in degrees C. When the weather indicates that it is raining outside, the surface convection heat transfer boundary conditions use the wetbulb temperature rather than the drybulb temperature (along with a high convection heat transfer coefficient). This override allows changing the outdoor wetbulb experienced by the surface exposed to exterior environment (when it is raining).
\item
  ``Outdoor Air Wind Speed.''~ This is the wind speed of the outdoor air exposed to the surface, in m/s. This is the velocity used for some surface convection heat transfer correlations for the outside-face of surfaces exposed to wind.
\item
  ``Outdoor Air Wind Direction.''~ This is the wind direction of the outdoor air exposed to the surface, in degree. This is the direction used for some surface convection heat transfer correlations for the outside-face of surfaces exposed to wind.
\item
  ``View Factor To Ground.''~ This is the view factor to ground of the outside face of the surface. This is the view factor used for ground diffuse solar and for radiant heat exchange. This is useful from the calling point named BeginZoneTimestepAfterInitHeatBalance and BeginZoneTimestepBeforeInitHeatBalance.
\end{itemize}

All of the above control types, except ``View Factor to Ground,'' are also available as actuators called ``Zone.''

Four actuators, called ``Other Side Boundary Conditions,'' are available for controlling the convection and radiation boundary conditions for surfaces that use ``OtherSideConditionsModel.''~ Each instance of a SurfaceProperty:OtherSideConditionsModel object will make available these actuators with the following four control types:

\begin{itemize}
\item
  ``Convection Bulk Air Temperature.''~ This is the temperature of the ambient air exposed to the surface, in degrees C.~ This is the temperature used for surface convection heat transfer boundary conditions on the outdoor, outside-face, other side of the surface.
\item
  ``Convection Heat Transfer Coefficient.''~ This is the heat transfer coefficient, in W/(m-K) used for surface convection boundary conditions on the outdoor, outside-face, or other side of the surface.
\item
  ``Radiation Effective Temperature.''~ This is the effective temperature of the environment surrounding the surface, in degrees C.~ This is the temperature used for surface thermal radiation heat transfer boundary conditions on the outdoor, outside-face, other side of the surface.
\item
  ``Radiation Linear Heat Transfer Coefficient.''~ This is the linearized heat transfer coefficient, in W/(m-K), used for surface thermal radiation boundary conditions on the outdoor, outside-face, or other side of the surface.
\end{itemize}

When using these actuators, values should be set for all four types.~ This boundary condition has no solar, only convection and radiation.

\subsection{Conduction Finite Difference}\label{conduction-finite-difference}

Several actuators, called ``CondFD Surface Material Layer,'' are available with actuated component control types for models that use the Conduction Finite Difference solution for conduction through surfaces. These are described below.

\begin{itemize}
    \item ``Thermal Conductivity''. Has units of [W/m-K]. Controls the thermal conductivity of the specified material layer within a given surface. Actuator is named from the surface name and material layer name combined with a colon ``:''. E.g. ``SURFNAME:MATERIALLAYERNAME''
    \item ``Specific Heat''. Has units of [J/kg-C]. Controls the specific heat for the specified material layer within a given surfaced. Actuator is named as described previously.
    \item ``Heat Flux''. Has units of [W/m2]. Controls the surface heat flux for the specified material layer within a given surface. The source is added directly after the material layer referenced and is not available for the final material layer. I.e. it must be inserted between two materal layers. Actuator is named as described previously. The actuator uses the same mechanism as the ``ConductionProperty:InternalHeatSource'' object, and is additive to the results computed by this object, if present. Actuator can only be used to add energy (i.e. positive numbers only), and the energy used is added to the electric heating sub-meter.
\end{itemize}

\subsubsection{Conduction Finite Difference Outputs}

\begin{itemize}
  \item Zone,Average,CondFD Phase Change Node Conductivity [W/m-K]
  \item Zone,Average,CondFD Phase Change Node Specific Heat [J/kg-K]
  \item Zone,Average,CondFD EMS Heat Source Power After Layer N [W]
  \item Zone,Average,CondFD EMS Heat Source Energy After Layer N [J]
\end{itemize}

\paragraph{CondFD Phase Change Node Conductivity}

This output reports thermal conductivity at the node level.

\paragraph{CondFD Phase Change Node Specific Heat}

This output reports specific heat at the node level.

\paragraph{CondFD EMS Heat Source Power After Layer N}

This output reports the heat power added after material layer N

\paragraph{CondFD EMS Heat Source Energy After Layer N}

This output reports the heat energy added after material layer N
