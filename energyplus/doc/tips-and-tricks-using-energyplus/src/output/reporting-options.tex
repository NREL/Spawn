\section{Reporting Options}\label{reporting-options}

There are many report variables in EnergyPlus. The ones available for a specific simulation are listed in the report data dictionary (rdd) file. These report variables may be generated automatically if the following is included in the input file.

\begin{lstlisting}

Output:VariableDictionary,
  Regular; !- Key Field
\end{lstlisting}

When the object above is included in an input file, the rdd file is available for review AFTER the simulation has completed. If this object is not included in the input file, the user may still use report variables, but must select them based on the objects included in the simulation. The Input Output Reference document describes all report variables available for each EnergyPlus object.

There are two flavors to output variables: ZONE or HVAC.~ ZONE does not mean that it is a zone variable -- rather, it is produced at the Zone Time Step (the same timestep that you specify in the Timestep object. HVAC type variables, likewise, are produced at the HVAC timestep (which can differ from the zone timestep frequency based on the ConvergenceLimits object).

There are several choices on format with this object. You can specify ``Regular'' as the key field and the rdd will show all report variables along with the variable description as shown below.

\begin{itemize}
\tightlist
\item
  HVAC,Average,Boiler Heating Output Rate {[}W{]}
\item
  HVAC,Sum,Boiler Heating Output Energy {[}J{]}
\item
  HVAC,Average,Boiler Gas Consumption Rate {[}W{]}
\item
  HVAC,Sum,Boiler Gas Consumption {[}J{]}
\item
  HVAC,Average,Boiler Water Inlet Temp {[}C{]}
\item
  HVAC,Average,Boiler Water Outlet Temp {[}C{]}
\item
  HVAC,Average,Boiler Water Mass Flow Rate {[}kg/s{]}
\item
  HVAC,Average,Boiler Parasitic Electric Consumption Rate {[}W{]}
\item
  HVAC,Sum,Boiler Parasitic Electric Consumption {[}J{]}
\item
  HVAC,Average,Boiler Part-Load Ratio {[}{]}
\end{itemize}

As an alternative, the key field ``IDF'' may also be used.

\begin{lstlisting}

Output:VariableDictionary,
  IDF; !- Key Field
\end{lstlisting}

With this option the rdd will format the report variable so that they may be copied directly into the input file using a text editor.

\begin{itemize}
\tightlist
\item
  Output:Variable,*,Boiler Heating Output Rate,hourly; !- HVAC Average {[}W{]}
\item
  Output:Variable,*,Boiler Heating Output Energy,hourly; !- HVAC Sum {[}J{]}
\item
  Output:Variable,*,Boiler Gas Consumption Rate,hourly; !- HVAC Average {[}W{]}
\item
  Output:Variable,*,Boiler Gas Consumption,hourly; !- HVAC Sum {[}J{]}
\item
  Output:Variable,*,Boiler Water Inlet Temp,hourly; !- HVAC Average {[}C{]}
\item
  Output:Variable,*,Boiler Water Outlet Temp,hourly; !- HVAC Average {[}C{]}
\item
  Output:Variable,*,Boiler Water Mass Flow Rate,hourly; !- HVAC Average {[}kg/s{]}
\item
  Output:Variable,*,Boiler Parasitic Electric Consumption Rate,hourly; !- HVAC Average {[}W{]}
\item
  Output:Variable,*,Boiler Parasitic Electric Consumption,hourly; !- HVAC Sum {[}J{]}
\item
  Output:Variable,*,Boiler Part-Load Ratio,hourly; !- HVAC Average {]}
\end{itemize}

A different version of the output variable object is shown below.

\begin{lstlisting}

Output:Variable,
  *, !- Key Value
  Boiler Heating Output Rate, !- Variable Name
  hourly, !- Reporting Frequency
  MyReportVarSchedule; !- Schedule Name

  Schedule:Compact,
  MyReportVarSchedule, !- Name
  On/Off, !- Schedule Type Limits Name
  Through: 1/20, !- Field 1
  For: AllDays, !- Field 2
  Until: 24:00, 0.0, !- Field 4
  Through: 12/31, !- Field 5
  For: AllDays, !- Field 6
  Until: 24:00, 1.0; !- Field 8

  ScheduleTypeLimits,
  On/Off, !- Name
  0:1, !- Range
  DISCRETE; !- Numeric Type
\end{lstlisting}

This allows several options for reporting. First the key value may be an asterisk (*) where all report variables of this type are reported (for all boilers). Or the key value could be specified such that only a single output will be generated. For example if the key value was specified as ``My Boiler'' and a boiler object with the name My Boiler was included in the input, only the Boiler Heating Output Rate for this specific boiler will be in the output file (.csv). The reporting output for all other boilers in the simulation will not be included in the csv file.

The reporting frequency is also another option and may be one of several choices (e.g., Timestep, Hourly, Daily, Monthly, RunPeriod, Environment, Annual or Detailed).

The detailed reporting frequency reports the data for every simulation time step (HVAC variable time steps). This choice is useful for detailed troubleshooting and reporting. The other choices average or sum the data over the selected interval. Timestep refers to the zone Timestep/Number of Timesteps in hour value and reports the data at regular intervals. Using RunPeriod, Environment, or Annual will have the same affect on the reporting frequency and refer to the length of the simulaiton as specified in the RunPeriod object.

\begin{lstlisting}

Timestep,
  4; !- Number of Timesteps per Hour


  RunPeriod,
  1, !- Begin Month
  1, !- Begin Day of Month
  12, !- End Month
  31, !- End Day of Month
  Tuesday, !- Day of Week for Start Day
  Yes, !- Use Weather File Holidays and Special Days
  Yes, !- Use Weather File Daylight Saving Period
  No, !- Apply Weekend Holiday Rule
  Yes, !- Use Weather File Rain Indicators
  Yes; !- Use Weather File Snow Indicator
\end{lstlisting}

A schedule may also be used to turn on or off report variable at selected intervals.

Table reports and meters are also available as reporting options. See the Input Output and Engineering Reference manuals for further details.
